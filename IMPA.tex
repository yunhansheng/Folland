\documentclass[hidelinks]{article}

\usepackage[a4paper,top=3cm,bottom=3cm,left=3cm,right=3cm,marginparwidth=1.75cm]{geometry}
\usepackage{amsmath,amsthm,amsfonts}
\usepackage[utf8]{inputenc}
\usepackage{amssymb}
\usepackage[mathscr]{eucal}
\usepackage{graphicx}
\usepackage{xcolor}
\usepackage{gensymb}
\usepackage{enumitem}

\setlength\parindent{0pt}

\newtheoremstyle{dotless}{}{}{\itshape}{}{\bfseries}{}{ }{}

\theoremstyle{definition}
\newtheorem*{defin}{DEF}
\newtheorem*{eg}{EXAMPLE}
\theoremstyle{dotless}
\newtheorem{proposition}{PROP}[section]
\newtheorem{claim}{CLM}[section]
\theoremstyle{remark}
\newtheorem*{remark}{Remark}

\usepackage{hyperref}
\hypersetup{colorlinks=false}



\begin{document}

\section*{\S0. Introduction: a non-measurable set}
\setcounter{section}{0}

We want to show that traditional notion of ``length" (``area", ``volume" etc.) are not well defined. \bigbreak
As our experience tells us, the traditional notion of ``length" on $\mathbb{R}$
\[
\sigma: \mathscr{P}(\mathbb{R})\to \mathbb{R}\cup\{\pm\infty\}
\]
should possess the following properties:
\begin{enumerate}[label=\arabic*\degree]
    \item $\sigma((a,b])=b-a$
    \item $\sigma(A+x)=\sigma(A)$
    \item $\sigma(\coprod\limits_jA_j)=\sum\limits_j\sigma(A_j)$
\end{enumerate}

\begin{claim}\label{CLM 0.1}
$E\subset F\Rightarrow \sigma(E)\leq \sigma(F)$
\end{claim}

\begin{claim}\label{CLM 0.2}
Define an equivalence relation $x\sim y$ in $\mathbb{R}$ if $y-x\in \mathbb{Q}$. Construct $\Omega\subset (0,1]$ by the axiom of choice: take one and only one element from each of the equivalence classes of $\mathbb{R}$ that is in $(0,1]$, then
\[\forall p\neq q\in \mathbb{Q}:(\Omega+p)\cup(\Omega+q)=\emptyset\]
\end{claim}
\textcolor{red}{Make use of the fact that $\Omega$ contains \emph{only one} element from each equivalence class}\bigbreak

Take $q\in R=\mathbb{Q}\cap(-1,1)$, then on one hand, by \hyperref[CLM 0.2]{Claim 0.2}
\[\Omega +q\subset(-1,2)\Rightarrow\sigma(\coprod_{q\in R}(\Omega+q))\leq 3\Rightarrow\coprod_{q\in R}(\Omega+q)=0\Rightarrow\sigma(\coprod_{q\in R}(\Omega+q))=0\]

but on the other, by \hyperref[CLM 0.1]{Claim 0.1}
\[(0,1)\subset\coprod_{q\in R}(\Omega+q)\Rightarrow 1\leq\sigma(\coprod_{q\in R}(\Omega+q))=0\]
\textcolor{red}{$\forall x\in (0,1)$ Consider $\alpha\in\Omega\cap[x]_\sim$}\\
Hence the contradiction arises. \bigbreak

Solution: change the domain of $\sigma$ from $\mathscr{P}(\mathbb{R})$ to a smaller class of subsets of $\mathbb{R}$.

\newpage
\section*{\S1. Classes of subsets (semi-álgebras, álgebras and $\sigma$-álgebras), and set functions}
\setcounter{section}{1}

\begin{defin}
\textbf{semi-álgebra} of set $\Omega$: $\mathscr{S}\subset\mathscr{P}(\Omega)$ s.t.
\begin{enumerate}[label=\arabic*\degree]
    \item $\Omega\in\mathscr{S}$
    \item closed under \emph{finite} intersection
    \item $\forall A\in\mathscr{S}\ \exists E_j\in\mathscr{S}:A^c=\coprod\limits_1^nE_j$
\end{enumerate}
\end{defin}

\begin{eg}
Let $\Omega=\mathbb{R}$, then $\mathscr{S}=\{\mathbb{R},\{(a,b]:a<b,a,b\in \mathbb{R}\},\{(a,\infty]:a\in \mathbb{R}\},\{(-\infty,b]:b\in\mathbb{R}\},\emptyset\}$ forms a semi-álgebra of $\Omega$
\end{eg}

\begin{defin}
\textbf{álgebra} of set $\Omega$:
$\mathscr{A}\subset\mathscr{P}(\Omega)$ s.t.
\begin{enumerate}[label=\arabic*\degree]
    \item $\Omega\in\mathscr{A}$
    \item closed under \emph{finite} intersection
    \item Closed under complement
\end{enumerate}
\end{defin}

\begin{defin}
\textbf{$\sigma$-álgebra} of set $\Omega$:
$\mathscr{F}\subset\mathscr{P}(\Omega)$ s.t.
\begin{enumerate}[label=\arabic*\degree]
    \item $\Omega\in\mathscr{F}$
    \item closed under \emph{countable} intersection
    \item Closed under complement
\end{enumerate}
\end{defin}

\begin{claim}
$\mathscr{F}\subset\mathscr{P}(\Omega)$, $\mathscr{F}$ is a $\sigma$-álgebra $\Rightarrow$ $\mathscr{F}$ is an álgebra $\Rightarrow$ $\mathscr{F}$ is a semi-álgebra
\end{claim}

\begin{defin}
\textbf{($\sigma$-)álgebra generated by $\mathscr{C}\subset\mathscr{P}(X)$}: ($\sigma$-)álgebra $\mathscr{A}(\mathscr{C})$ s.t.
\begin{enumerate}[label=\arabic*\degree]
    \item $\mathscr{C}\subset\mathscr{A}(\mathscr{C})$
    \item ($\sigma$-)álgebra $\mathscr{B}\supset\mathscr{C}\Rightarrow\mathscr{B}\supset\mathscr{A}(\mathscr{C})$
\end{enumerate}
or more explicitly written, the intersection of all ($\sigma$-)álgebras $\mathscr{A}_\alpha$ that contains $\mathscr{C}$
\[\mathscr{A}(\mathscr{C})=\bigcap\limits_\alpha\mathscr{A}_\alpha\]
\end{defin}

\begin{proposition}\label{Prop 1.1}
Let $\mathscr{S}$ be a semi-álgebra on set $\Omega$ and $\mathscr{A}(\mathscr{S})$ the álgebra generated by it, then
\[\forall A\in\mathscr{A}(\mathscr{S})\ \exists E_j\in\mathscr{S}:A=\coprod_1^nE_j \]
\end{proposition}

\begin{remark}
This is a very special property that only \emph{álgebra} generated by semi-álgebra has, not shared by \emph{$\sigma$-álgebra}.
\end{remark}

\begin{defin}
Let $\mathscr{C}\subset\mathscr{P}(\Omega)$ and $\emptyset\in\mathscr{C}$, a function $f:\mathscr{C}\to\mathbb{R}_+\cup\{\pm\infty\}$ is \textbf{additive} if
\begin{enumerate}[label=\arabic*\degree]
    \item $f(\emptyset)=0$
    \item $E_j,E=\coprod\limits_jE_j\in\mathscr{C}\Rightarrow f(E)=\sum\limits_jf(E_j)$
\end{enumerate} for finite $j$s, and \textbf{$\sigma-$additive} for countable $j$s
\end{defin}

\begin{claim}
If $\exists\ C\in\mathscr{C}:f(C)<+\infty$, then $2\degree$ implies $1\degree$, in the above definition.
\end{claim}

\begin{eg}~
\begin{enumerate}[label=\arabic*\degree]
    \item discrete measure: Let $\Omega$ be a set and $\mathscr{C}\subset\mathscr{P}(\Omega)$. Let $\{x_j\}$ and $\{p_j\}$ be two sequences on $\Omega$ and $\mathbb{R}_{0+}$ resp., then the discrete measure $\mu$ on $\mathscr{C}$ by $\mu:A\mapsto\sum_jp_j\{x_j\in A\}$ is $\sigma$-additive
    \item Let $\Omega=(0,1)$ and $\mathscr{C}=\{(a,b]:0\leq a<b<1\}\subset\mathscr{P}(\Omega)$, then $\mu:\mathscr{C}\to\mathbb{R}_+\cup\{+\infty\}$ defined by
    \[\mu(a,b]=
    \begin{cases} 
      +\infty & a=0 \\
      b-a & a<b
   \end{cases}
\]
is additive but \emph{not} $\sigma$-additive, since for a positive strictly decreasing sequence $\{x_j\}$ that converges to 0
\[+\infty=\mu(0,0.5]\neq\mu(\sum_j(x_{j+1},x_j])=0.5\]
\end{enumerate}
\end{eg}

\begin{defin}
\textbf{measure} on semi-álgebra $\mathscr{S}$: $\mu:\mathscr{S}\to\mathbb{R}_+\cup\{+\infty\}$ s.t. \begin{enumerate}[label=\arabic*\degree]
    \item $\mu$ is $\sigma$-additive 
    \item $\exists S\in\mathscr{S}:\mu(S)<+\infty$
\end{enumerate}
\end{defin}

\newpage
\section*{\S2. Set functions}
\setcounter{section}{2}

\begin{defin}
Let $\mathscr{C}\subset\mathscr{P}(\Omega)$, then function $f:\mathscr{C}\to\mathbb{R}_+\cup\{+\infty\}$ is \textbf{continuous from below (resp. above)} at $E\in\mathscr{C}$ if $\forall E_j\uparrow E\textrm{ (resp. }\forall E_j\downarrow E)$ in $\mathscr{C}$, or
\begin{enumerate}[label=\arabic*\degree]
    \item $E_j\subset E_{j+1}$ (resp. $E_j\supset E_{j+1}$)
    \item $\bigcup\limits_jE_j=E$ (resp. $\bigcap\limits_jE_j=E$)
    \item ($\exists j_0:f(E_{j_0})<+\infty$)
\end{enumerate}
then $\lim f(E_j)=f(E)$
\end{defin}

\begin{remark}
Consider $E_n=[n,+\infty)\in\mathscr{P}(\mathbb{R})$ and $f$ the function of "length", then $E_n\downarrow\emptyset$ yet $f(E_n)=+\infty\neq0$. Hence we need condition 3\degree for above-continuity.
\end{remark}

\begin{defin}
Let $\mathscr{C}\subset\mathscr{P}(\Omega)$, then $f:\mathscr{C}\to\mathbb{R}_+\cup\{+\infty\}$ is \textbf{continuous} at $E\in\mathscr{C}$ if it is both continuous from above and below
\end{defin}

\begin{proposition}
Let $\mathscr{A}\subset\mathscr{P}(\Omega)$ be an álgebra on set $\Omega$ and $\mu:\mathscr{A}\to\mathbb{R}_+\cup\{+\infty\}$ an additive function on $\mathscr{A}$, then \begin{enumerate}[label=\arabic*\degree]
    \item $\mu$ is $\sigma$-additive $\Rightarrow$ $\forall E\in\mathscr{A}:\mu\textrm{ is continuous at }E$
    \item $\mu$ is continuous from below $\Rightarrow$ $\mu$ is $\sigma$-additive
    \item $\mu$ is continuous from above at $\emptyset$ and $\mu$ is finite $\Rightarrow$ $\mu$ is $\sigma$-additive
\end{enumerate}
\end{proposition}
\textcolor{red}{In 1\degree  use the usual trick to construct disjoint set-sequence from set-sequence, and pay attention to the finiteness in proving the "continuous above" part; In 2\degree construct $F_n$ to be the disjoint union of first $n$ disjoint terms; 3\degree follows similarly, pay attention to how the finiteness of $\mu$ is used to guarantee the finiteness in the definition 3\degree of continuous from below}

\begin{proposition}
Let $\mathscr{S}\subset\mathscr{P}(\Omega)$ be an semi-álgebra, then $(\sigma\textrm{-})$additive function $\mu: \mathscr{S}\to \mathbb{R}_+\cup\{+\infty\}$ can be extended to a unique function $\nu: \mathscr{A}(\mathscr{S})\to \mathbb{R}_+\cup\{+\infty\}$ on the álgebra generated by $\mathscr{S}$, i.e.
\begin{enumerate}[label=\arabic*\degree]
    \item $\nu$ is $(\sigma)$-additive
    \item $\forall S\in\mathscr{S}:\nu(S)=\mu(S)$
    \item $\forall S\in\mathscr{S}:\nu_1(S)=\nu_2(S)$ $\Rightarrow$ $\forall A\in\mathscr{A}(\mathscr{S}):\nu_1(S)=\nu_2(S)$
\end{enumerate}
\end{proposition}
\textcolor{red}{Additive: use \hyperref[Prop 1.1]{Prop 1.1}, then $\forall A\in\mathscr{A}(\mathscr{S})\ \exists E_j\in\mathscr{S}:A=\coprod_{j=1}^nE_j$. Construct $\nu:A\mapsto\coprod_{j=1}^n\mu(E_j)$, and first show that this map is well-defined (use additivity of $\mu$). The rest then follows trivially.\smallbreak
$(\sigma\textrm{-})$additive:
\[\nu(A)=\sum_{j=1}^n\nu(E_j)=\sum_{j=1}^n\sum_{k\geq1}\sum_{l=1}^{m_k}\mu(E_j\cap E_{k,l})=\sum_{k\geq1}\sum_{l=1}^{m_k}\nu(E_{k,l})=\sum_{k\geq1}\nu(A_k)\]
In the second and third equality use the trick $A\subset B\Rightarrow A=A\cap B$}

\begin{remark}
Note how only \emph{álgebra} generated by a semi-álgebra can extend functions, while $\sigma$-álgebra generated by a semi-álgebra does not work.
\end{remark}

\newpage
\section*{\S3. Carathéodory theorem}
\setcounter{section}{3}
In this section, $\mathscr{S}$ is a semi-álgebra of set $\Omega$, $\nu$ is the unique $(\sigma\textrm{-})$additive extension of measure $\mu:\mathscr{S}\to\mathbb{R}_+\cup\{+\infty\}$ on the álgebra $\mathscr{A}=\mathscr{A}(\mathscr{S})$ generated by the semi-álgebra $\mathscr{S}$.

\begin{defin}
Let $\mathscr{C}\subset\mathscr{P}(\Omega)$ and $\emptyset\in\mathscr{C}$, an \textbf{outer measure} on $\mathscr{C}$: $\mu:\mathscr{C}\to\mathbb{R}_+\cup\{+\infty\}$ s.t.\begin{enumerate}[label=\arabic*\degree]
    \item $\mu(\emptyset)=0$
    \item $E\subset F\Rightarrow\mu(E)\leq\mu(F)$
    \item $E_i\in\mathscr{C}\Rightarrow\mu(\bigcup\limits_iE_i)\leq\sum\limits_i\mu(E_i)$
\end{enumerate}
\end{defin}

Construct
\[\pi^*:\mathscr{P}(\Omega)\to\mathbb{R}_+\cup\{+\infty\}\textrm{ by }\pi^*:A\mapsto\inf\limits_{\{E_i\}}\sum\limits_{i\geq1}\nu(E_i)\]
for ${E_i}\in\mathscr{A}$ is a covering of $A\in\mathscr{P}(\Omega)$ ($A\subset\bigcup\limits_{i\geq1}E_i$)

\begin{claim}\label{CLM 3.5}
$\pi^*$ is an outer measure
\end{claim}
\textcolor{red}{1\degree and 2\degree are trivial; for 3\degree the $\pi^*(E_i)=\infty$ case is trivial, assume $\pi^*(E_i)<\infty$. $\forall\epsilon>0$ consider a covering $\{H_{i,k}\}\in\mathscr{A}$ of $E_i$ s.t.
\[\pi^*(E_i)\leq\sum_{k\geq1}\nu(H_{i,k})\leq\pi^*(E_i)+\frac{\epsilon}{2^i}\]
then since $\bigcup\limits_i\bigcup\limits_{k\geq1}H_{i,k}$ is a covering of $E=\bigcup\limits_iE_i$
\[\pi^*(E)\leq\sum\limits_{i,k}\nu(H_{i,k})\leq\sum\limits_i(\pi^*(E_i)+\frac{\epsilon}{2^i})=\sum\limits_i\pi^*(E_i)+\epsilon\]}

\begin{defin}
$A$ is \textbf{$\pi^*$-measurable} ($A\in\mathscr{M}$) if $\forall E\in\mathscr{P}(\Omega):\pi^*(E)=\pi^*(E\cap A)+\pi^*(E\cap A^c)$
\end{defin}

\begin{claim}\label{CLM 3.6}
$\forall E\in\mathscr{P}(\Omega):\pi^*(E)\leq\pi^*(E\cap A)+\pi^*(E\cap A^c)$ always holds from subadditivity of $\pi^*$
\end{claim}

\begin{claim}
$\mathscr{M}$ is a $\sigma$-álgebra and $\mathscr{A}\subset\mathscr{M}$, it then follows that the $\sigma$-álgebra generated $\mathscr{F}(\mathscr{A})\subset\mathscr{M}$
\end{claim}
\textcolor{red}{$\mathscr{A}\subset\mathscr{M}$: same method of proving \hyperref[CLM 3.5]{CLM 3.5}\newline
$\mathscr{M}$ is a $\sigma$-álgebra: 1\degree and 3\degree in the def. of $\sigma$-álgebra are trivial, and 2\degree can be proved in 2 steps \begin{enumerate}[label=\arabic*\degree]
    \item finite union: let $A,B\in\mathscr{M}$, observe first that
    \[E\cap(A\cup B)=\{[E\cap(A\cup B)]\cap A\}\cup\{[E\cap(A\cup B)]\cap A^c\}=[E\cap A]\cup[(E\cap A^c)\cap B]\] then by subadditivity
    \begin{equation*}
    \begin{split}
        \pi^*(E)=\pi^*(E\cap A)+\pi^*(E\cap A^c)& =\pi^*(E\cap A)+\pi^*[(E\cap A^c)\cap B]+\pi^*[(E\cap A^c)\cap B^c] \\ & \geq\pi^*[E\cap(A\cup B)]+\pi^*[E\cap(A\cup B)^c]
    \end{split}
    \end{equation*}
    hence $A\cup B\in\mathscr{M}$ by \hyperref[CLM 3.6]{CLM 3.6}
    \item infinite union: let $A_j\in\mathscr{M}$ and $A=\bigcup_{j\geq1}A_j$, then
    \begin{equation*}
    \begin{split}
        \pi^*(E)&=\pi[E\cap(\bigcup_{j=1}^nA_j)]+\pi^*[E\setminus(\bigcup\limits_{j=1}^nA_j)]\\ &\geq\pi^*[E\cap(\bigcup_{j=1}^nA_j)]+\pi^*(E\setminus A)\\
        &=\pi^*[E\cap(\coprod_{j=1}^nF_j)]+\pi^*(E\setminus A)=\sum_{j=1}^n\pi^*(E\cap F_j)+\pi^*(E\setminus A)
    \end{split} 
    \end{equation*}
    where $F_j$ are the disjoint sets constructed from $A_j$ with the same limit.\newline
    Take limit $n\to\infty$
    \begin{equation*}
    \begin{split}
        \pi^*(E)&\geq\sum_{j\geq1}\pi^*(E\cap F_j)+\pi^*(E\setminus A)\\
        &\geq\pi^*[E\cap(\bigcup_{j\geq1}F_j)]+\pi^*(E\setminus A)=\pi^*(E\cap A)+\pi^*(E\setminus A)
    \end{split}
    \end{equation*}
    hence $A=\cup_{j\geq1}A_j\in\mathscr{M}$ by \hyperref[CLM 3.6]{CLM 3.6}
\end{enumerate}}





\newpage
\section*{\S4. Monotone classes}
\setcounter{section}{4}

\end{document}