\documentclass[hidelinks]{article}

\usepackage[a4paper,top=3cm,bottom=3cm,left=3cm,right=3cm,marginparwidth=1.75cm]{geometry}
\usepackage{amsmath,amsthm,amsfonts}
\usepackage[utf8]{inputenc}
\usepackage{amssymb}
\usepackage[mathscr]{eucal}
\usepackage{graphicx}
\usepackage{xcolor}
\usepackage{gensymb}
\usepackage{marvosym}
\usepackage{enumitem}
\usepackage[toc]{multitoc}
\renewcommand*{\multicolumntoc}{2}
\setlength{\columnseprule}{1pt}

\setlength\parindent{0pt}

\newtheoremstyle{dotless}{}{}{\itshape}{}{\bfseries}{}{ }{}

\theoremstyle{definition}
\newtheorem*{defin}{DEF}
\newtheorem*{eg}{EXAMPLE}
\theoremstyle{dotless}
\newtheorem{proposition}{PROP}[section]
\newtheorem{claim}{CLM}[section]
\theoremstyle{remark}
\newtheorem*{remark}{Remark}

\usepackage{hyperref}
\hypersetup{colorlinks=false}



\begin{document}

\begin{titlepage}
   \begin{center}
      \LARGE\textbf{Measure and Integration: An Introduction}\medbreak
      \Large\textsl{A. Scrjabin}\bigbreak
      \centerline{\href{https://en.wikipedia.org/wiki/The_Five_Senses_(series)}{\includegraphics[width=0.6\linewidth]{ee.jpg}}}
      \normalsize{Think allegorically, like people in the Middle Ages!}
      \end{center}
  \tableofcontents
   \bigbreak\bigbreak
   Disclaimer: These are the lecture notes I complied while watching Professor Claudio Landim's \href{https://www.youtube.com/playlist?list=PLo4jXE-LdDTQq8ZyA8F8reSQHej3F6RFX}{\textbf{Lectures on Measure Theory}} on YouTube, under the channel Instituto Nacional de Matemática Pura e Aplicada. Professor Landim does a wonderful job explaining this beautiful topic. In support of the free-culture movement, please feel free to use and reproduce this note if needed, though I would appreciate if you indicate the source. I used the online \LaTeX\  editor \href{https://www.overleaf.com}{\textbf{Overleaf} to write this note.} This is my first \LaTeX\ math note.
\end{titlepage}

\section*{\S0. Introduction}
\setcounter{section}{0}
\addcontentsline{toc}{section}{\protect\numberline{}\S0. Introduction}

We want to show that traditional notion of ``length" (``area", ``volume" etc.) are not well defined. \bigbreak
As our experience tells us, the traditional notion of ``length" on $\mathbb{R}$
\[
\sigma: \mathscr{P}(\mathbb{R})\to \mathbb{R}\cup\{\pm\infty\}
\]
should possess the following properties:
\begin{enumerate}[label=\arabic*\degree]
    \item $\sigma((a,b])=b-a$
    \item $\sigma(A+x)=\sigma(A)$
    \item $\sigma(\coprod\limits_jA_j)=\sum\limits_j\sigma(A_j)$
\end{enumerate}

\begin{claim}\label{CLM 0.1}
$E\subset F\Rightarrow \sigma(E)\leq \sigma(F)$
\end{claim}

\begin{claim}\label{CLM 0.2}
Define an equivalence relation $x\sim y$ in $\mathbb{R}$ if $y-x\in \mathbb{Q}$. Construct $\Omega\subset (0,1]$ by the axiom of choice: take one and only one element from each of the equivalence classes of $\mathbb{R}$ that is in $(0,1]$, then
\[\forall p\neq q\in \mathbb{Q}:(\Omega+p)\cup(\Omega+q)=\emptyset\]
\end{claim}
\textcolor{red}{Make use of the fact that $\Omega$ contains \emph{only one} element from each equivalence class}\bigbreak

Take $q\in R=\mathbb{Q}\cap(-1,1)$, then on one hand, by \hyperref[CLM 0.2]{Claim 0.2}
\[\Omega +q\subset(-1,2)\Rightarrow\sigma(\coprod_{q\in R}(\Omega+q))\leq 3\Rightarrow\coprod_{q\in R}(\Omega+q)=0\Rightarrow\sigma(\coprod_{q\in R}(\Omega+q))=0\]

but on the other, by \hyperref[CLM 0.1]{Claim 0.1}
\[(0,1)\subset\coprod_{q\in R}(\Omega+q)\Rightarrow 1\leq\sigma(\coprod_{q\in R}(\Omega+q))=0\]
\textcolor{red}{$\forall x\in (0,1)$ Consider $\alpha\in\Omega\cap[x]_\sim$}\\
Hence the contradiction arises. \bigbreak

Solution: change the domain of $\sigma$ from $\mathscr{P}(\mathbb{R})$ to a smaller class of subsets of $\mathbb{R}$.

\bigbreak

\section*{\S1. Classes of subsets}
\setcounter{section}{1}
\addcontentsline{toc}{section}{\protect\numberline{}\S1. Classes of subsets}

\begin{defin}
\textbf{semi-álgebra} of set $\Omega$: $\mathscr{S}\subset\mathscr{P}(\Omega)$ s.t.
\begin{enumerate}[label=\arabic*\degree]
    \item $\Omega\in\mathscr{S}$
    \item closed under \emph{finite} intersection
    \item $\forall A\in\mathscr{S}\ \exists E_j\in\mathscr{S}:A^c=\coprod\limits_1^nE_j$
\end{enumerate}
\end{defin}

\begin{eg}
Let $\Omega=\mathbb{R}$, then $\mathscr{S}=\{\mathbb{R},\{(a,b]:a<b,a,b\in \mathbb{R}\},\{(a,\infty]:a\in \mathbb{R}\},\{(-\infty,b]:b\in\mathbb{R}\},\emptyset\}$ forms a semi-álgebra of $\Omega$
\end{eg}

\begin{defin}
\textbf{álgebra} of set $\Omega$:
$\mathscr{A}\subset\mathscr{P}(\Omega)$ s.t.
\begin{enumerate}[label=\arabic*\degree]
    \item $\Omega\in\mathscr{A}$
    \item closed under \emph{finite} intersection
    \item closed under complement
\end{enumerate}
\end{defin}

\begin{defin}
\textbf{$\sigma$-álgebra} of set $\Omega$:
$\mathscr{F}\subset\mathscr{P}(\Omega)$ s.t.
\begin{enumerate}[label=\arabic*\degree]
    \item $\Omega\in\mathscr{F}$
    \item closed under \emph{countable} intersection
    \item closed under complement
\end{enumerate}
\end{defin}

\begin{claim}
$\mathscr{F}\subset\mathscr{P}(\Omega)$, $\mathscr{F}$ is a $\sigma$-álgebra $\Rightarrow$ $\mathscr{F}$ is an álgebra $\Rightarrow$ $\mathscr{F}$ is a semi-álgebra
\end{claim}

\begin{defin}
\textbf{($\sigma$-)álgebra generated by $\mathscr{C}\subset\mathscr{P}(X)$}: ($\sigma$-)álgebra $\mathscr{A}(\mathscr{C})$ s.t.
\begin{enumerate}[label=\arabic*\degree]
    \item $\mathscr{C}\subset\mathscr{A}(\mathscr{C})$
    \item ($\sigma$-)álgebra $\mathscr{B}\supset\mathscr{C}\Rightarrow\mathscr{B}\supset\mathscr{A}(\mathscr{C})$
\end{enumerate}
or more explicitly written, the intersection of all ($\sigma$-)álgebras $\mathscr{A}_\alpha$ that contains $\mathscr{C}$
\[\mathscr{A}(\mathscr{C})=\bigcap\limits_\alpha\mathscr{A}_\alpha\]
\end{defin}

\begin{proposition}\label{Prop 1.1}
Let $\mathscr{S}$ be a semi-álgebra on set $\Omega$ and $\mathscr{A}(\mathscr{S})$ the álgebra generated by it, then
\[\forall A\in\mathscr{A}(\mathscr{S})\ \exists E_j\in\mathscr{S}:A=\coprod_1^nE_j \]
\end{proposition}

\begin{remark}
This is a very special property that only \emph{álgebra} generated by semi-álgebra has, not shared by \emph{$\sigma$-álgebra}.
\end{remark}

\begin{defin}
Let $\mathscr{C}\subset\mathscr{P}(\Omega)$ and $\emptyset\in\mathscr{C}$, a function $f:\mathscr{C}\to\mathbb{R}_+\cup\{\pm\infty\}$ is \textbf{additive} if
\begin{enumerate}[label=\arabic*\degree]
    \item $f(\emptyset)=0$
    \item $E_j,E=\coprod\limits_jE_j\in\mathscr{C}\Rightarrow f(E)=\sum\limits_jf(E_j)$
\end{enumerate} for finite $j$s, and \textbf{$\sigma-$additive} for countable $j$s
\end{defin}

\begin{claim}
If $\exists\ C\in\mathscr{C}:f(C)<+\infty$, then $2\degree$ implies $1\degree$, in the above definition.
\end{claim}

\begin{eg}~
\begin{enumerate}[label=\arabic*\degree]
    \item \textbf{discrete measure}: Let $\Omega$ be a set and $\mathscr{C}\subset\mathscr{P}(\Omega)$. Let $\{x_j\}$ and $\{p_j\}$ be two sequences on $\Omega$ and $\mathbb{R}_{0+}$ resp., then the discrete measure $\mu$ on $\mathscr{C}$ by $\mu:A\mapsto\sum_jp_j1_{\{x_j\in A\}}$ is $\sigma$-additive
    \item Let $\Omega=(0,1)$ and $\mathscr{C}=\{(a,b]:0\leq a<b<1\}\subset\mathscr{P}(\Omega)$, then $\mu:\mathscr{C}\to\mathbb{R}_+\cup\{+\infty\}$ defined by
    \[\mu(a,b]=
    \begin{cases} 
      +\infty & a=0 \\
      b-a & a<b
   \end{cases}
\]
is additive but \emph{not} $\sigma$-additive, since for a positive strictly decreasing sequence $\{x_j\}$ that converges to 0
\[+\infty=\mu(0,0.5]\neq\mu(\sum_j(x_{j+1},x_j])=0.5\]
\end{enumerate}
\end{eg}

\begin{defin}
\textbf{measure} on semi-álgebra $\mathscr{S}$: $\mu:\mathscr{S}\to\mathbb{R}_+\cup\{+\infty\}$ s.t. \begin{enumerate}[label=\arabic*\degree]
    \item $\mu$ is $\sigma$-additive 
    \item $\exists S\in\mathscr{S}:\mu(S)<+\infty$
\end{enumerate}
\end{defin}

\bigbreak

\section*{\S2. Set functions}
\setcounter{section}{2}
\addcontentsline{toc}{section}{\protect\numberline{}\S2. Set functions}

\begin{defin}
Let $\mathscr{C}\subset\mathscr{P}(\Omega)$, then function $f:\mathscr{C}\to\mathbb{R}_+\cup\{+\infty\}$ is \textbf{continuous from below (resp. above)} at $E\in\mathscr{C}$ if $\forall E_j\uparrow E\textrm{ (resp. }\forall E_j\downarrow E)$ in $\mathscr{C}$, or
\begin{enumerate}[label=\arabic*\degree]
    \item $E_j\subset E_{j+1}$ (resp. $E_j\supset E_{j+1}$)
    \item $\bigcup\limits_jE_j=E$ (resp. $\bigcap\limits_jE_j=E$)
    \item ($\exists j_0:f(E_{j_0})<+\infty$)
\end{enumerate}
then $\lim f(E_j)=f(E)$
\end{defin}

\begin{remark}
Consider $E_n=[n,+\infty)\in\mathscr{P}(\mathbb{R})$ and $f$ the function of "length", then $E_n\downarrow\emptyset$ yet $f(E_n)=+\infty\neq0$. Hence we need condition 3\degree for above-continuity.
\end{remark}

\begin{defin}
Let $\mathscr{C}\subset\mathscr{P}(\Omega)$, then $f:\mathscr{C}\to\mathbb{R}_+\cup\{+\infty\}$ is \textbf{continuous} at $E\in\mathscr{C}$ if it is both continuous from above and below
\end{defin}

\begin{proposition}\label{Prop 2.2}
Let $\mathscr{A}\subset\mathscr{P}(\Omega)$ be an álgebra on set $\Omega$ and $\mu:\mathscr{A}\to\mathbb{R}_+\cup\{+\infty\}$ an additive function on $\mathscr{A}$, then \begin{enumerate}[label=\arabic*\degree]
    \item $\mu$ is $\sigma$-additive $\Rightarrow$ $\forall E\in\mathscr{A}:\mu\textrm{ is continuous at }E$
    \item $\mu$ is continuous from below $\Rightarrow$ $\mu$ is $\sigma$-additive
    \item $\mu$ is continuous from above at $\emptyset$ and $\mu$ is finite $\Rightarrow$ $\mu$ is $\sigma$-additive
\end{enumerate}
\end{proposition}
\textcolor{red}{In 1\degree  use the usual trick to construct disjoint set-sequence from set-sequence, and pay attention to the finiteness in proving the "continuous above" part; In 2\degree construct $F_n$ to be the disjoint union of first $n$ disjoint terms; 3\degree follows similarly, pay attention to how the finiteness of $\mu$ is used to guarantee the finiteness in the definition 3\degree of continuous from below}

\begin{proposition}\label{Prop 2.3}
Let $\mathscr{S}\subset\mathscr{P}(\Omega)$ be an semi-álgebra, then $(\sigma\textrm{-})$additive function $\mu: \mathscr{S}\to \mathbb{R}_+\cup\{+\infty\}$ can be extended to a unique function $\nu: \mathscr{A}(\mathscr{S})\to \mathbb{R}_+\cup\{+\infty\}$ on the álgebra generated by $\mathscr{S}$, i.e.
\begin{enumerate}[label=\arabic*\degree]
    \item $\nu$ is $(\sigma)$-additive
    \item $\forall S\in\mathscr{S}:\nu(S)=\mu(S)$
    \item $\forall S\in\mathscr{S}:\nu_1(S)=\nu_2(S)$ $\Rightarrow$ $\forall A\in\mathscr{A}(\mathscr{S}):\nu_1(S)=\nu_2(S)$
\end{enumerate}
\end{proposition}
\textcolor{red}{Additive: use \hyperref[Prop 1.1]{Prop 1.1}, then $\forall A\in\mathscr{A}(\mathscr{S})\ \exists E_j\in\mathscr{S}:A=\coprod_{j=1}^nE_j$. Construct $\nu:A\mapsto\coprod_{j=1}^n\mu(E_j)$, and first show that this map is well-defined (use additivity of $\mu$). The rest then follows trivially.\smallbreak
$(\sigma\textrm{-})$additive:
\[\nu(A)=\sum_{j=1}^n\nu(E_j)=\sum_{j=1}^n\sum_{k\geq1}\sum_{l=1}^{m_k}\mu(E_j\cap E_{k,l})=\sum_{k\geq1}\sum_{l=1}^{m_k}\nu(E_{k,l})=\sum_{k\geq1}\nu(A_k)\]
In the second and third equality use the trick $A\subset B\Rightarrow A=A\cap B$}

\begin{remark}
Note how only \emph{álgebra} generated by a semi-álgebra can extend functions, while $\sigma$-álgebra generated by a semi-álgebra does not work.
\end{remark}

\bigbreak

\section*{\S3. Carathéodory theorem}
\setcounter{section}{3}
\addcontentsline{toc}{section}{\protect\numberline{}\S3. Carathéodory theorem}

In this section, $\mathscr{S}$ is a semi-álgebra of set $\Omega$, $\nu$ is the unique $\sigma$-additive extension of measure $\mu:\mathscr{S}\to\mathbb{R}_+\cup\{+\infty\}$ on the álgebra $\mathscr{A}=\mathscr{A}(\mathscr{S})$ generated by the semi-álgebra $\mathscr{S}$.

\begin{defin}
Let $\mathscr{C}\subset\mathscr{P}(\Omega)$ and $\emptyset\in\mathscr{C}$, an \textbf{outer measure} on $\mathscr{C}$: $\mu:\mathscr{C}\to\mathbb{R}_+\cup\{+\infty\}$ s.t.\begin{enumerate}[label=\arabic*\degree]
    \item $\mu(\emptyset)=0$
    \item $E\subset F\Rightarrow\mu(E)\leq\mu(F)$
    \item $E_i\in\mathscr{C}\Rightarrow\mu(\bigcup\limits_iE_i)\leq\sum\limits_i\mu(E_i)$
\end{enumerate}
\end{defin}

Construct
\[\pi^*:\mathscr{P}(\Omega)\to\mathbb{R}_+\cup\{+\infty\}\textrm{ by }\pi^*:A\mapsto\inf\limits_{\{E_i\}}\sum\limits_{i\geq1}\nu(E_i)\]
for ${E_i}\in\mathscr{A}$ is a covering of $A\in\mathscr{P}(\Omega)$ ($A\subset\bigcup\limits_{i\geq1}E_i$)

\begin{claim}\label{CLM 3.5}
$\pi^*$ is an outer measure
\end{claim}
\textcolor{red}{1\degree and 2\degree are trivial; for 3\degree the $\pi^*(E_i)=\infty$ case is trivial, assume $\pi^*(E_i)<\infty$. $\forall\epsilon>0$ consider a covering $\{H_{i,k}\}\in\mathscr{A}$ of $E_i$ s.t.
\[\pi^*(E_i)\leq\sum_{k\geq1}\nu(H_{i,k})\leq\pi^*(E_i)+\frac{\epsilon}{2^i}\]
then since $\bigcup\limits_i\bigcup\limits_{k\geq1}H_{i,k}$ is a covering of $E=\bigcup\limits_iE_i$
\[\pi^*(E)\leq\sum\limits_{i,k}\nu(H_{i,k})\leq\sum\limits_i(\pi^*(E_i)+\frac{\epsilon}{2^i})=\sum\limits_i\pi^*(E_i)+\epsilon\]}

\begin{defin}
$A$ is \textbf{$\pi^*$-measurable} ($A\in\mathscr{M}$) if $\forall E\in\mathscr{P}(\Omega):\pi^*(E)=\pi^*(E\cap A)+\pi^*(E\cap A^c)$
\end{defin}

\begin{claim}\label{CLM 3.6}
$\forall E\in\mathscr{P}(\Omega):\pi^*(E)\leq\pi^*(E\cap A)+\pi^*(E\cap A^c)$ always holds from subadditivity of $\pi^*$
\end{claim}

\begin{claim}
$\mathscr{M}$ is a $\sigma$-álgebra and $\mathscr{A}\subset\mathscr{M}$, it then follows that the $\sigma$-álgebra generated $\mathscr{F}(\mathscr{A})\subset\mathscr{M}$
\end{claim}
\textcolor{red}{$\mathscr{A}\subset\mathscr{M}$: same method of proving \hyperref[CLM 3.5]{CLM 3.5}\newline
$\mathscr{M}$ is a $\sigma$-álgebra: 1\degree and 3\degree in the def. of $\sigma$-álgebra are trivial, and 2\degree can be proved in 2 steps \begin{enumerate}[label=\arabic*\degree]
    \item finite union: let $A,B\in\mathscr{M}$, observe first that
    \[E\cap(A\cup B)=\{[E\cap(A\cup B)]\cap A\}\cup\{[E\cap(A\cup B)]\cap A^c\}=[E\cap A]\cup[(E\cap A^c)\cap B]\] then by subadditivity
    \begin{equation*}
    \begin{split}
        \pi^*(E)=\pi^*(E\cap A)+\pi^*(E\cap A^c)& =\pi^*(E\cap A)+\pi^*[(E\cap A^c)\cap B]+\pi^*[(E\cap A^c)\cap B^c] \\ & \geq\pi^*[E\cap(A\cup B)]+\pi^*[E\cap(A\cup B)^c]
    \end{split}
    \end{equation*}
    hence $A\cup B\in\mathscr{M}$ by \hyperref[CLM 3.6]{CLM 3.6}
    \item infinite union: let $A_j\in\mathscr{M}$ and $A=\bigcup_{j\geq1}A_j$, then
    \begin{equation*}
    \begin{split}
        \pi^*(E)&=\pi[E\cap(\bigcup_{j=1}^nA_j)]+\pi^*[E\setminus(\bigcup\limits_{j=1}^nA_j)]\\ &\geq\pi^*[E\cap(\bigcup_{j=1}^nA_j)]+\pi^*(E\setminus A)\\
        &=\pi^*[E\cap(\coprod_{j=1}^nF_j)]+\pi^*(E\setminus A)=\sum_{j=1}^n\pi^*(E\cap F_j)+\pi^*(E\setminus A)
    \end{split} 
    \end{equation*}
    where $F_j$ are the disjoint sets constructed from $A_j$ with the same limit.\newline
    Take limit $n\to\infty$
    \begin{equation*}
    \begin{split}
        \pi^*(E)&\geq\sum_{j\geq1}\pi^*(E\cap F_j)+\pi^*(E\setminus A)\\
        &\geq\pi^*[E\cap(\bigcup_{j\geq1}F_j)]+\pi^*(E\setminus A)=\pi^*(E\cap A)+\pi^*(E\setminus A)
    \end{split}
    \end{equation*}
    hence $A=\cup_{j\geq1}A_j\in\mathscr{M}$ by \hyperref[CLM 3.6]{CLM 3.6}
\end{enumerate}}

\begin{proposition}(Carathéodory's extension theorem) The restriction $\pi^*|_\mathscr{M}$ on the collection of $\sigma$-measurable sets $\mathscr{M}$ is a $\sigma$-additive extension of $\nu$ on an álgebra $\mathscr{A}$.
\end{proposition}
\textcolor{red}{$\pi^*(A)=\nu(A)$ on $\mathscr{A}$:\newline
the $\leq$ part is trivial; the $\geq$ part let $\{E_j\}\in\mathscr{A}$ be a covering of $A$ and construct from $E_j$ disjoint unions $\{F_j\}\in\mathscr{A}$ with same limit $A$, then
\[\nu(A)=\nu(\coprod_jF_j\cap A)=\sum_j\nu(F_j\cap A)\leq\sum_j\nu(E_j)\]
since the choice of the covering is arbitrary, $\nu(A)\leq\inf\sum\limits_j\nu(E_j)=\pi^*(A)$\newline
$\pi^*(\coprod\limits_jA_j)=\sum\limits_j\pi^*(A_j)$:\newline
the $\leq$ part follows from subadditivity of $\pi^*$, the $\geq$ part follows the finite case by taking $n\to\infty$}

\begin{defin}
$\Omega$ is \textbf{$\sigma$-finite} for $\mu$ if $\exists \{E_j\}\in\Omega\ \forall j:(E_j\uparrow\Omega)\wedge(\mu(E_j)<\infty)$
\end{defin}

\begin{defin}
$\mathscr{G}\subset\mathscr{P}(\Omega)$ is a \textbf{monotone class} if
\begin{enumerate}[label=\arabic*\degree]
    \item $(\{A_j\}\in\mathscr{G})\wedge(A_j\subset A{j+1})\Rightarrow A=\bigcup\limits_{j\geq1}A_j\in\mathscr{G}$
    \item $(\{B_j\}\in\mathscr{G})\wedge(B_j\supset B{j+1})\Rightarrow B=\bigcap\limits_{j\geq1}B_j\in\mathscr{G}$
\end{enumerate}
\end{defin}

\begin{claim}
Let $\{\mathscr{G}_\alpha\}_{\alpha\in I}$ be a collection of monotone classes, then $\bigcap\limits_{\alpha\in I}\mathscr{G}_\alpha$ is a monotone class
\end{claim}

\begin{defin}
Monotone class \textbf{generated by} $\mathscr{C}\subset\mathscr{P}(\Omega)$ is \[\forall\mathscr{G}_\alpha\supset\mathscr{C}:\mathscr{G}(\mathscr{C})=\bigcap\limits_\alpha\mathscr{G}_\alpha\]
\end{defin}

\begin{proposition}\label{Prop 3.5}
Let $\mathscr{F}(\mathscr{A})$ be the sigma-álgebra generated by álgebra $\mathscr{A}$ and $\mu_1,\mu_2:\mathscr{F}(\mathscr{A})\to\mathbb{R}_+\cup\{+\infty\}$, $\Omega$ is $\sigma$-finite for $\mu_1$, then
\[\mu_1|_\mathscr{A}=\mu_2|_\mathscr{A}\Rightarrow\mu_1=\mu_2\]
\end{proposition}
\textcolor{red}{Consider
\[\mathscr{B}_n=\{E\in\mathscr{F}(\mathscr{A}):\mu_1(E\cap E_n)=\mu_2(E\cap E_n)\}\subset\mathscr{F}(\mathscr{A})\]
By $\sigma$-additivity of $\mu$ and \hyperref[Prop 2.2]{Prop 2.2}, $\mathscr{B}_n$ is a monotone class. Since $A\subset\mathscr{B}_n$, by \hyperref[Prop 4.6]{Prop 4.6}, $\mathscr{B}_n\supset\mathscr{F}(\mathscr{A})$. Hence $\mathscr{B}_n=\mathscr{F}(\mathscr{A})$.}

\bigbreak

\section*{\S4. Monotone classes}
\setcounter{section}{4}
\addcontentsline{toc}{section}{\protect\numberline{}\S4. Monotone classes}

\begin{proposition}\label{Prop 4.6}
For álgebra $\mathscr{A}$, $\sigma$-álgebra generated by $\mathscr{A}$ equals the monotone class generated by $\mathscr{A}$, or \[\mathscr{F}(\mathscr{A})=\mathscr{M}(\mathscr{A})\]
\end{proposition}
\textcolor{red}{The direction $\mathscr{M}(\mathscr{A})\subset\mathscr{F}(\mathscr{A})$ is trivial, to prove the other direction, we only need to show that $\mathscr{M}(\mathscr{A})$ is a $\sigma$-álgebra, and since $\mathscr{M}(\mathscr{A})$ is a monotone class, it follows from $\mathscr{M}(\mathscr{A})$ is an álgebra which will be proved in steps: construct for $E\in\mathscr{M}(\mathscr{A})$
\[\mathscr{G}_E=\{F\in\mathscr{M}(\mathscr{A}):E\setminus F,E\cap F,F\setminus E\subset\mathscr{M}(\mathscr{A})\}\]
\begin{enumerate}[label=\arabic*\degree]
    \item $\forall E\in\mathscr{A}:\mathscr{M}(\mathscr{A})\subset\mathscr{G}_E$
    \item $\forall E\in\mathscr{M}(\mathscr{A}):\mathscr{M}(\mathscr{A})\subset\mathscr{G}_E$
    \item $\mathscr{M}(\mathscr{A})$ is an álgebra
\end{enumerate}}

\bigbreak

\section*{\S5. The Lebesgue measure}
\setcounter{section}{5}
\addcontentsline{toc}{section}{\protect\numberline{}\S5. The Lebesgue measure}

Construct Lebesgue measure: let semi-álgebra \[\mathscr{S}=\{\emptyset,\mathbb{R},(a,b],(a,\infty),(-\infty,b]\}\] and additive $\mu:\mathscr{S}\to\overline{\mathbb{R}_+}$ by the ``usual" definition of ``length on the real line":\begin{enumerate}[label=\arabic*\degree]
    \item $\mu(\emptyset)=0$
    \item $\mu((a,b])=b-a$
    \item $\mu((-\infty,b],(a,\infty),\mathbb{R})=0$
\end{enumerate}
By \hyperref[Prop 2.3]{Prop 2.3} $\mu$ extends uniquely to additive $\nu:\mathscr{A}(\mathscr{S})\to\overline{\mathbb{R}_+}$

\begin{proposition}
$\mu$ defined above is $\sigma$-additive, i.e.
\[\forall A_j,A=\coprod_{j\geq1}A_j\in\mathscr{S}:\mu(A)=\sum_{j\geq1}\mu(A_j)\]
\end{proposition}
\textcolor{red}{Consider the extension $\nu$, then trivially $\nu(A)\geq\sum_{j\geq1}\mu(A_j)$; the other direction can be proved 2 steps:
\begin{enumerate}[label=\arabic*\degree]
    \item Let $A=(a,b]$ and $A_j=(a_j,b_j]$, then $\forall\epsilon>0$
    \[[a+\epsilon,b]\subset(a,b]=\coprod_{j\geq1}(a_j,b_j]\subset\bigcup_{j\geq1}(a_j,b_j+\frac{\epsilon}{2^j})\]
    By usual topology on $\mathbb{R}$
    \[(a+\epsilon,b]\subset[a+\epsilon,b]\subset\sum_{k=1}^m(a_{j_k},b_{j_k}+\frac{\epsilon}{2^{j_k}})\subset\sum_{k=1}^m(a_{j_k},b_{j_k}+\frac{\epsilon}{2^{j_k}}]\]
    Hence
    \[b-a-\epsilon=\nu((a+\epsilon,b])\leq\sum_{k=1}^mb_{j_k}-a_{j_k}+\frac{\epsilon}{2^{j_k}}\leq\sum_{k\geq1}b_j-a_j+\frac{\epsilon}{2^j}=\epsilon+\sum_{j\geq1}b_j-a_j\]
    \item Let $E_n=(-n,n]\uparrow\mathbb{R}$ and any $A_j,A=\coprod_{j\geq1}A_j\in\mathscr{S}$, then by the previous step
    \[\nu(A\cap E_n)=\sum_{j\geq1}\nu(A_j\cap E_j)\leq\sum_{j\geq1}\nu(A_j)\]
    Take $n\to\infty$
    \[\lim\nu(A\cap E_n)=\nu(A)\leq\sum_{j\geq1}\nu(A_j)\]
\end{enumerate}\bigbreak
Alternatively, one can prove by using \hyperref[Prop 2.2]{Prop 2.2}: for $\{E_k\}\in\mathscr{A}$ s.t. $E_k\downarrow\emptyset$ and $E_k\in[-N,N]$, we show $\lim\nu(E_k)=0$\smallbreak
Assume otherwise that $\exists\delta>0:0<2\delta\leq\nu(E_k)$\newline
Let
\[E_1=\coprod\limits_{j=1}^{m_1}(a_{1_j},b_{1_j}],F_1=\coprod\limits_{j=1}^{m_1}(a_{1_j}+\epsilon_1,b_{1_j}]\]
for some $\epsilon>0$, then $F_1\subset\overline{F_1}=G_1\subset E_1$\newline
Choose $\epsilon_1>0$ s.t. $\nu(E_1\setminus F_1)\leq\frac{\delta}{2}$\medbreak
Observe that $E_2\cap F_1\neq\emptyset$, indeed
\[0<2\delta-\frac{\delta}{2}\leq\nu(E_2)-\nu(E_1\setminus F_1)\leq\nu(E_2)-\nu(E_2\setminus F_1)=\nu(E_2\cap E_1)\]
Let
\[E_2\cap F_1=\coprod\limits_{j=1}^{m_2}(a_{1_j},b_{1_j}],F_2=\coprod\limits_{j=1}^{m_2}(a_{1_j}+\epsilon_2,b_{1_j}]\]
for some $\epsilon>0$, then $F_2\subset\overline{F_2}=G_2\subset E_2\cap F_1\subset F_1\subset G_1$\newline
Choose $\epsilon_2>0$ s.t. $\nu(E_2\cap F_1)\setminus F_2)\leq\frac{\delta}{4}$\newline
then
\[\nu(E_2\setminus F_2)=\nu((E_2\cap E_1)\setminus F_2)+\nu((E_2\setminus E_1)\setminus F_2)\leq\frac{\delta}{4}+\nu(E_2\setminus E_1)\leq\frac{\delta}{4}+\frac{\delta}{2}\]
Construct inductively $F_n$, and obtain a countable decreasing sequence of compact sets $\{G_k\}$ s.t. $G_k\subset E_k$ and $G_k\neq\emptyset$, yet $\bigcap\limits_{k\geq1}G_k\subset\bigcap\limits_{k\geq1}E_k=\emptyset$, which by basic topology on $\mathbb{R}$ \Lightning}
\bigbreak
Hence by the Carathéodory's extension theorem and \hyperref[Prop 3.5]{Prop 3.5}, $\nu$ can be further extended uniquely on the $\sigma$-álgebra generated by $\mathscr{A}$. Note that $\Omega=\mathbb{R}$ meets the $\sigma$-finiteness condition of \hyperref[Prop 3.5]{Prop 3.5}.

\bigbreak

\section*{\S6. Complete measures}
\setcounter{section}{6}
\addcontentsline{toc}{section}{\protect\numberline{}\S6. Complete measures}

\begin{defin}
Let $\mathscr{C}\subset\mathscr{P}(\Omega)$ and $\mu:\mathscr{C}\to\overline{\mathbb{R}_+}$, then $(\mu,\mathscr{C})$ is \textbf{complete} (or $\mathscr{C}$ is \textbf{$\mu$-complete}) if $A\in\mathscr{C}$, $\mu(A)=0$, and $E\subset A$ implies $E\in\mathscr{C}$. These $E$s are called \textbf{negligible sets}
\end{defin}

\begin{claim}
Let $\mathscr{F}\subset\mathscr{P}(\Omega)$ be a $\sigma$-álgebra and $\mu:\mathscr{F}\to\overline{\mathbb{R}_+}$ a measure, then
\[\overline{\mathscr{F}}=\{A\cup N:A\in\mathscr{F},N\textrm{ are negligible sets}\}\]
is a $\sigma$-álgebra.
\end{claim}
\textcolor{red}{1\degree and 3\degree are trivial, for 2\degree let $A=E\cup N$, $E\in\mathscr{F}$, $N\subset H\in\mathscr{F}$, and $\mu(H)=0$
\[A^c=(E\cup N)^c=[(E\cup N)^c\cup H]\cup[(E\cup N)^c\cup H^c]\]
where $(E\cup N)^c\cup H\in H$, and $(E\cup N)^c\cup H^c\in\mathscr{F}$}

\begin{claim}
The natural extension of $\mu$ from $\mathscr{F}$ to $\overline{\mathscr{F}}$ by $\overline{\mu}:A\cup N\mapsto\mu(A)$ is $\sigma$-additive and unique.
\end{claim}

\begin{proposition}
Measure space $(\Omega,\overline{\mathscr{F}},\overline{\mu})$ is complete
\end{proposition}
\textcolor{red}{For $A\subset E\in\overline{\mathscr{F}}$ and $\overline{\mu}(E)=0$, rewrite $A=\emptyset\cup A$, and prove $\overline{\mu}(A)=0$}

\begin{claim}
Let $\mathscr{M}$ be the collection of $\sigma$-measurable sets and $\pi^*:\mathscr{M}\to\overline{\mathbb{R}_+}$, then $\mathscr{M}$ is $\pi^*$-complete.
\end{claim}
\textcolor{red}{Let $A\subset B\in\mathscr{M}$ and $\pi^*(B)=0$. To prove $A\in\mathscr{M}$, by \hyperref[CLM 3.6]{CLM 3.6} we only need to prove \[\pi^*(E)\geq\pi^*(E\cap A)+\pi^*(E\cap A^c)\]
and on one hand,
\[E\cap A\subset A\subset B\Rightarrow\pi^*(E\cap A)\leq\pi^*(B)=0\]
On the other,
\[\pi^*(E\cap A^c)\leq\pi^*(E)\]}

\begin{remark}
In particular, the Lebesgue measure is complete.
\end{remark}

\bigbreak

\section*{\S7. Approximation theorems}
\setcounter{section}{7}
\addcontentsline{toc}{section}{\protect\numberline{}\S7. Approximation theorems}

\begin{proposition} (Approximation theorem)
Let $\mathscr{A}\subset\mathscr{P}(\Omega)$ be an álgebra and $\mathscr{F}=\mathscr{F}(\mathscr{A})$ the $\sigma$-álgebra generated by $\mathscr{A}$. Let $\mu:\mathscr{F}\to\overline{\mathbb{R}_+}$ be a measure and $A\in\mathscr{F}$ s.t. $\mu(A)<\infty$, then
\[\forall\epsilon>0\ \exists E\in\mathscr{A}:\mu(E\setminus A)+\mu(A\setminus E)<\epsilon\]
\end{proposition}
\textcolor{red}{Recall that $\forall A\in\mathscr{F}:\mu(A)<\infty$
\[\pi^*(A)=\mu(A)=\inf_{\{A_i\}}\sum_{i\geq1}\nu(A_i)<\infty\]
for all coverings $\{A_i\}\in\mathscr{A}$ and measure $\nu$ on $\mathscr{A}$. Hence
\[\forall\epsilon>0\ \exists\{A_i\}\in\mathscr{A}:\pi^*(A)\leq\sum_{i\geq1}\nu(A_i)\leq\pi^*(A)+\epsilon\]
Hence $\exists n_0\in\mathbb{N}:\sum\limits_{i\geq n_0}\nu(A_i)\leq\epsilon$, and let $E=\bigcup\limits_{i=1}^{n_0}A_i\in\mathscr{A}$.}

\begin{remark}
In fact, if we further impose that $\Omega$ is $\sigma$-finite for $\mu$, then the same procedure can approximate elements in $\overline{\mathscr{F}}$ by elements in $\mathscr{A}$.
\end{remark}

\begin{defin}
\textbf{Borel $\sigma$-álgebra} $\mathscr{B}\in\mathscr{P}(\Omega)$: the smallest $\sigma$-álgebra that contains all open sets in $\Omega$
\end{defin}

\begin{defin}
Let $\mathscr{B}\subset\mathscr{F}$, then measure $\mu:\mathscr{F}\to\overline{\mathbb{R}_+}$ is \textbf{regular} if
\[\forall A\in\mathscr{F}\ \forall\epsilon>0\ \exists F\subset A\subset G:\mu(G\setminus F)<\epsilon\]
where $F$ and $G$ are closed and open sets respectively
\end{defin}

\begin{claim}
$\mu$ is regular $\Rightarrow$ $\mathscr{F}\subset\overline{\mathscr{B}}_\mu$
\end{claim}
\textcolor{red}{Common trick to construct sequences $\{G_i\}$ and $\{F_j\}$ s.t. $n\mu(G_i\setminus F_j)<1{}$}

\begin{claim}
Let $\mu$ be the Lebesgue measure, then the collection of all $\mu$-measurable sets $\mathscr{L}$ is a $\sigma$-álgebra, and $\mathscr{L}$ is thus called the \textbf{Lebesgue $\sigma$-álgebra}.
\end{claim}

\begin{proposition}
The Lebesgue measure $\mu:\mathscr{L}\to\overline{\mathbb{R}}_+$ is regular.
\end{proposition}
\textcolor{red}{Let $A\in\mathscr{L}$ and $E_n=[-n,n]$, then $\mu(A_n)=\mu(A\cap E_n)<\infty$. Hence \begin{equation}\label{eq:1}
\forall\epsilon>0\ \exists\{B_{n,k}\}_{k\geq1}\in\mathscr{A}\textrm{ a cover of }A_n:\mu(A_n)\leq\sum_{k\geq1}\mu(B_{n,k})\leq\mu(A_n)+\epsilon\end{equation}
Construct open sets $G_{n,k}$: by \hyperref[Prop 1.1]{Prop 1.1}
\[B_{n,k}\in\mathscr{A}\Rightarrow\exists\delta_{n,k,j}>0:B_{n,k}=\coprod_{j=1}^{l_{n,k}}(a_{n,k,j},b_{n,k,j}]\subset\bigcup_{j=1}^{l_{n,k}}(a_{n,k,j},b_{n,k,j}+\delta_{n,k,j})=G_{n,k}\]
Choose $\delta_{n,k,j}$ s.t. \begin{equation}\label{eq:2}
\sum_{j=1}^{l_{m,k}}<\frac{\epsilon}{2^n2^k}\Rightarrow\mu(B_{n,k})\leq\mu(G_{n,k})\leq\mu(B_{n,k})+\frac{\epsilon}{2^n2^k}\end{equation}
Since $A_n\subset\bigcup_{k\geq1}B_{n,k}\subset\bigcup_{k\geq1}G_{n,k}=G_n$, by equations \ref{eq:1} and \ref{eq:2}
\[\mu(G_n)\leq\sum_{k\geq1}\mu(G_{n,k})\leq\sum_{k\geq1}\mu(B_{n,k})+\frac{\epsilon}{2^n}\leq\mu(A_n)+\frac{\epsilon}{2^{n-1}}\]
Let open set $G=\bigcup_{n\geq1}G_n$, then $A\subset G$, and
\[\mu(G\setminus A)=\mu(\bigcup_{n\geq1}G_n\setminus \bigcup_{n\geq1}A_n)=\mu(\bigcup_{n\geq1}(G_n\setminus A_n))\leq\bigcup_{n\geq1}\mu(G_n\setminus A_n)\leq\sum_{n\geq1}\frac{\epsilon}{2^{n-1}}=2\epsilon\]
For the other part, by what we proved already $\exists\textrm{ open set }H:\mu(H\setminus A^c)\leq\epsilon$, then one can easily check that slosed set $F=H^c\subset A$ and $\mu(A\setminus F)\leq\epsilon$.}

\bigbreak

\section*{\S8. Measurable functions}
\setcounter{section}{8}
\addcontentsline{toc}{section}{\protect\numberline{}\S8. Measurable functions}

\begin{defin}
Let measure space $(\Omega,\mathscr{F},\mu)$, $\mathscr{B}$ the Borel $\sigma$-álgebra, and the \textbf{extended Borel $\sigma$-álgebra} \[\overline{\mathscr{B}}=\{A\cup B:A\in\mathscr{B},B\in\mathscr{P}(\{\pm\infty\})\}\]
then $f:\Omega\to\overline{\mathbb{R}}$ is \textbf{$\mathscr{F}$-measurable} if $\forall A\in\overline{\mathscr{B}}:f^{-1}(A)\in\mathscr{F}$
\end{defin}

\begin{claim}
$\overline{\mathscr{B}}$ defined above is a $\sigma$-álgebra.
\end{claim}

\begin{claim}\label{CLM 8.15}
Let measure space $(\Omega,\mathscr{F},\mu)$ and function $f:\Omega\to\overline{\mathbb{R}}$, then $f$ is $\mathscr{F}$-measurable iff \[\forall x\in\mathbb{R}:f^{-1}((-\infty,x])\in\mathscr{F}\]
\end{claim}
\textcolor{red}{One direction is trivial, for the other, let
\[\mathscr{C}=\{A\in\overline{\mathscr{B}}:f^{-1}\in\mathscr{F}\}\subset\overline{\mathscr{B}}\]
We want to show that $\mathscr{C}\supset\overline{\mathscr{B}}$, and we only need to prove that $\mathscr{C}$ is a $\sigma$-álgebra. Since then
\[\mathscr{C}\supset\mathscr{F}(\{(-\infty,x]:x\in\mathbb{R}\})\supset\overline{\mathscr{B}}\]}

\begin{eg}~\\
\textbf{simple function}: Let $(\Omega,\mathscr{F},\mu)$ be a measure space, $E_1,E_2,...,E_n\in\mathscr{F}$ a finite \emph{disjoint} partition of $\Omega$, and $c_1,c_2,...,c_n\in\mathbb{R}$ a finite sequence. Then the simple function $f:\Omega\to\mathbb{R}$ by \[f:x\mapsto\sum_{j=1}^nc_j1_{\{x\in E_j\}}\]
is $\mathscr{F}$-measurable.
\end{eg}

\begin{defin}
Let $f$ be a positive simple function (meaning that $c_j$ defined above is non-negative), then the \textbf{integral of simple function} $f$:
\[I(f)=\sum_{j=1}^nc_j\mu(E_j)\]
\end{defin}

\begin{remark}
Integral of simple functions can already solve problems that Riemann integral is unable to solve. Consider the Dirichlet function on $(0,1]$, which is not Riemann-integrable, yet by the definition above the integral of this simple function does exist.
\end{remark}

\begin{claim}\label{CLM 8.16}
For simple functions $f=\sum\limits_{j=1}^mc_j1_{\{x\in E_j\}},g=\sum\limits_{k=1}^nd_j1_{\{x\in F_j\}}$
\[f\leq g\Rightarrow I(f)\leq I(g)\]
\end{claim}
\textcolor{red}{In the case of $\exists j,k:E_j\cap E_k\neq\emptyset$, $f\leq g$ forces $c_j\leq d_k$; in the other case $I(f)=I(g)=0$}

\begin{claim}
Let $(\Omega,\mathscr{F},\mu)$ is a measure space, and $g,f:\Omega\to\overline{\mathbb{R}}$ be $\mathscr{F}$-measurable functions, then those functions are also $\mathscr{F}$-measurable: $\forall\alpha>0$
\[\alpha+f,\alpha f+g,fg,1/f,f^+,|f|\]
\end{claim}
\textcolor{red}{Take good use of \hyperref[CLM 8.15]{CLM 8.15}; tricky equalities:
\[f^+=\max\{f,0\},f^-=\max\{-f,0\},|f|=f^++f^-\]
\[\{\omega\in\Omega:f(\omega)+g(\omega)<x\}=\bigcup_{r\in\mathbb{Q}}(\{\omega:f(\omega)<r\}\cup\{\omega:g(\omega)<x-r\})\in\mathscr{F}\]}

\begin{claim}
Let $(\Omega,\mathscr{F},\mu)$ be a measure space, and $\{f_n:\Omega\to\overline{\mathbb{R}}\}_{n\geq1}$ a sequence of $\mathscr{F}$-measurable functions, then these functions are also $\mathscr{F}$-measurable:
\[\sup_nf_n,\inf_nf_n,\lim\sup_nf_n,\lim\inf_nf_n,\lim f_n\]
Note that these operators are taken pointwise.
\end{claim}
\textcolor{red}{Since $\forall\omega\in\Omega\ \forall x\in\overline{\mathbb{R}}:f_n(\omega)>x\Rightarrow\sup_nf_n(\omega)>x$
\[\forall\omega\in\Omega\ \forall x\in\overline{\mathbb{R}}:\{\sup_nf_n(\omega)>x\}=\bigcup_{n\geq1}\{f_n(\omega)>x\}\]
Also note that $\inf\limits_nf_n=-\sup\limits_n\{-f_n\}$, and $\lim\sup\limits_nf_n=\inf\limits_k\sup\limits_{n\geq k}f_n$}

\begin{claim}
Construct measure space $(\Omega,\mathscr{F},\mu)$ where $\Omega$ is a topological space and $\mathscr{F}\supset\mathscr{B}$, let $f:\Omega\to\overline{\mathbb{R}}$, then
\begin{center}$f$ is continuous $\Rightarrow$ $f$ is $\mathscr{F}$-measurable\end{center}
\end{claim}

\begin{defin}
Let $(\Omega,\mathscr{F},\mu)$ be a measure space, and $\mathbb{P}$ a property of $f:\Omega\to\overline{\mathbb{R}}$, then $\mathbb{P}$ holds \textbf{almost surely} if $\mathbb{P}$ holds on $E$ for $E\in\mathscr{F}$ and $\mu(E^c)=0$
\end{defin}

\begin{proposition}
Let $(\mathbb{R},\mathscr{L},\lambda$ be a measure space ($\lambda$ is the Lebesgue measure), and $f,g:\mathbb{R}\to\overline{\mathbb{R}}$. Let $f$ be $\mathscr{L}$-measurable, then $g=f$ almost surely $\Rightarrow$ $g$ is $\mathscr{L}$-measurable.
\end{proposition}
\textcolor{red}{Let $g=f$ almost surely holds on $E\in\mathscr{L}$, then
\[\forall A\in\overline{\mathscr{B}}:\{\omega\in\Omega:g(\omega)\in A\}=(\{\omega\in\Omega:f(\omega)\in A\}\cap E)cup(\{\omega\in\Omega:g(\omega)\in A\}\cap E^c)\in\mathscr{L}\]}

\begin{remark}
The same statement does \emph{not} hold for the Borel $\sigma$-álgebra.
\end{remark}

\begin{claim}
Let $(\Omega,\mathscr{F},\mu)$ be a measure space, $f:\Omega\to\mathbb{R}$ be $\mathscr{F}$-measurable, and $g:\mathbb{R}\to\mathbb{R}$ be $\overline{\mathscr{B}}$-measurable, then $g\circ f:\Omega\to\mathbb{R}$ is $\mathscr{F}$-measurable.
\end{claim}

\bigbreak

\section*{\S9. Definition of the integral}
\setcounter{section}{9}
\addcontentsline{toc}{section}{\protect\numberline{}\S9. Definition of the integral}

\begin{claim}\label{CLM 9.21}
Let $f\geq0$ be $\mathscr{F}$-measurable, then $\exists$ monotonically increasing sequence of simple non-negative functions $\{f_n\}$ s.t. pointwise $\lim\limits_nf_n=f$
\end{claim}
\textcolor{red}{(Stepwise approximation) construct $f_n$:
\[f_n(x)=\sum_{k=0}^{n2^n-1}\frac{k}{2^n}1_{\{\frac{k}{2^n}\leq f(x)<\frac{k+1}{2^n}\}}+n1_{\{f(x)\geq n\}}=\begin{cases}
n & f(x)\geq n\\
\frac{k}{2^n} & \frac{k}{2^n}\leq f(x)<\frac{k+1}{2^n}
\end{cases}\]
for $k\in[0,n2^n-1]\cap\mathbb{N}$\newline
Since
\[\{\frac{k}{2^n}\leq f(x)<\frac{k+1}{2^n}\}=f^{-1}([\frac{k}{2^n},\frac{k+1}{2^n})),\{f(x)\geq n\}=f^{-1}([n,\infty))\in\mathscr{F}\]
$f_n$ is indeed simple and non-negative. Then think geometrically how $\lim_nf_n$ approximates $f$. Lastly show $f_n\leq f_{n+1}$ by discussing $f_n(x)\geq n+1$,$n\leq f_n(x)<n+1$,$f(x)<n$ three cases.}

\begin{claim}\label{CLM 9.22}
Let $\{f_n\}$ be a sequence of non-negative, monotonically increasing simple functions, and $g\leq\lim\limits_nf_n$ a non-negative simple function, then
\[I(g)\leq\lim\limits_nI(f_n)\]
\end{claim}
\textcolor{red}{The proof follows in two steps:\begin{enumerate}[label=\arabic*\degree]
    \item Let $g=c1_E$ for $E\in\mathscr{F}$ and $c\geq0$. Since $c=0\Rightarrow I(g)=0\leq\lim\limits_nI(f_n)$, WLOG $c>0$.\newline
    Construct $A_n=\{x\in E:f_n(x)\geq c-\epsilon\}$, then $A_n\subset A_{n+1}$ and $\bigcup\limits_{n\geq1}A_n=E$. We have
    \[(c-\epsilon)\mu(A_n)=I((c-\epsilon)1_{A_n})\leq I(f_n1_{A_n})\leq I(f_n)\]
    Take limits on both sides, and since $\mu$ is continuous at $E$ from below
    \[I(g)-\epsilon\mu(E)=(c-\epsilon)\lim_n\mu(A_n)\leq\lim_nI(f_n)\]
    Hence $I(g)\leq\lim\limits_nI(f_n)$ follows from the arbitrariness of $\epsilon>0$.
    \item Let $g=\sum\limits_{k=1}^nc_k1_{E_k}$ for $c_k\geq0$ and $E_k\in\mathcal{F}$ forms a partition of $\Omega$. Then
    \[I(g)=\sum_{k=1}^nI(c_k1_{E_k})\leq\sum_{k=1}^n\lim_nI(f_n1_{E_k})=\lim_nI(\sum_{k=1}^nf_n1_{E_k})=\lim_nI(f_n)\]
\end{enumerate}}

\begin{proposition}
Let $\{f_n\},\{g_n\}$ be sequences of non-negative, monotonically increasing simple functions that pointwise converges to $f$, then
\[\lim_nI(f_n)=\lim_nI(g_n)\]
\end{proposition}
\textcolor{red}{Follows from \hyperref[CLM 9.22]{CLM 9.22}}

\begin{defin}
The \textbf{integral} of a $\mathscr{F}$-measurable \emph{non-negative} function $f$: \[I(f)=\lim I(f_n)\]
for $\{f_n\}$ a monotonically increasing sequence of non-negative simple functions that (pointwise) converges to $f$.\newline
A $\mathscr{F}$-measurable function $f$ is \textbf{integrable} if both $I(f^+),I(f^-)<\infty$, and the \textbf{integral} of it is
\[I(f)=I(f^+)-I(f^-)\]
\end{defin}

\bigbreak

\section*{\S10. Properties of the integral}
\setcounter{section}{10}
\addcontentsline{toc}{section}{\protect\numberline{}\S10. Properties of the integral}

\begin{claim}
Let $f,g$ be non-negative $\mathscr{F}$-measurable functions, then
\[\forall k\geq0:I(kf+g)=kI(f)+I(g)\]
\end{claim}
\textcolor{red}{Prove first for $f,g$ are simple functions, then the general statement follows by approximation.\newline
If $f=\sum\limits_jc_j1_{E_j},g=\sum\limits_kd_k1_{f_k}$ for $c_j,d_k\geq0$ and $E_j,F_k\in\mathscr{F}$ partitions of $\Omega$, then
\[E_j=\sum_kE_j\cap F_k\Rightarrow1_{E_j}=\sum_k1_{E_j\cap F_k}\Rightarrow f=\sum_{j,k}c_j1_{E_j\cap F_k}\]
Similarly, $g=\sum\limits_{j,k}d_k1_{E_j\cap F_k}$}

\begin{claim}
Let $f$ be a $\mathscr{F}$-measurable function and $A\in\mathscr{F}$, then $f$ is integrable $\Rightarrow$ $1_Af$ is integrable.
\end{claim}

\bigbreak

\section*{\S11. Convergence of integrals}
\setcounter{section}{11}
\addcontentsline{toc}{section}{\protect\numberline{}\S11. Convergence of integrals}

\bigbreak

\section*{\S12. Product measures}
\setcounter{section}{12}
\addcontentsline{toc}{section}{\protect\numberline{}\S12. Product measures}

\bigbreak

\section*{\S13. Fubini's theorem}
\setcounter{section}{13}
\addcontentsline{toc}{section}{\protect\numberline{}\S13. Fubini's theorem}

\end{document}