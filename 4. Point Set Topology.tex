\documentclass[hidelinks]{article}

\usepackage{xeCJK}
\usepackage[a4paper,top=3cm,bottom=3cm,left=3cm,right=3cm,marginparwidth=1.75cm]{geometry}
\usepackage{amsmath,amsthm,amsfonts}
\usepackage[utf8]{inputenc}
\usepackage{amssymb}
\usepackage[mathscr]{eucal}
\usepackage{graphicx}

\setcounter{section}{+3}
\setcounter{subsection}{+3}

\theoremstyle{definition}
\newtheorem*{defin}{Def}
\theoremstyle{plain}
\newtheorem{theorem}{Thm}[section]
\newtheorem{proposition}[theorem]{Prop}
\theoremstyle{remark}
\newtheorem*{remark}{Remark}

\usepackage{hyperref}
\hypersetup{colorlinks=false}



\begin{document}

\section{Point Set Topology}

\subsection{Topological Spaces}
\begin{defin}~\\
topology on $X$: $P(X)$的子集,包含$\varnothing$和$X$,且在任意并、有限交运算下封闭 \newline
discrete topology: $P(X)$ \newline
trivial (indiscrete) topology: $\{\varnothing,X\}$ \newline
cofinite topology: $\{U\subset X\mid U=\varnothing \textrm{ or } U^c \textrm{ is finite}\}$ \newline
relative topology on $Y\subset X$ induced by topology $\mathscr{T}$ on $X$:
$$\mathscr{T}_Y:=\{U\cap Y\mid U\in \mathscr{T}\}$$
open set: members of $\mathscr{T}$ \newline
relative open: members of the induced topology \newline
neighborhood of $x\in X$ ($E\subset X$): $A\subset X$ s.t. $x\in A^o$ ($E\subset A^o$) \newline
accumulation point $x$ of $A$: $A\cap (U\setminus\{x\})\neq \varnothing$ for every neighborhood $U$ of $x$
\end{defin}

\begin{remark}
The concept of closed set, interior, closure, boundary, (nowhere) dense, separable and convergence are consistent with that in the context of metric spaces (metric topology).
\end{remark}

\begin{proposition}
$\overline{A}=A\cup acc(A)$, and that A为闭集 \textrm{iff} $acc(A)\subset A$
\end{proposition}

\begin{defin}~\\
若$\mathscr{T}_1\subset\mathscr{T}_2$,则$\mathscr{T}_1$ is weaker (coarser) than $\mathscr{T}_2$, or $\mathscr{T}_2$ is stronger (finer) than $\mathscr{T}_1$ \newline
topology generated by $\mathscr{E}\subset P(X)$:
$$\mathscr{T}(\mathscr{E}):= \textrm{intersection of all topologies on X containing }\mathscr{E}$$
$\mathscr{E}$ is a subbase for $\mathscr{T}(\mathscr{E})$ \newline
neighborhood base for $\mathscr{T}$ on $X$ at $x\in X$: $\mathscr{N}\subset \mathscr{T}$ s.t.
$$(x\in X \textrm{ for all } V\in \mathscr{N})\wedge(\exists V\in \mathscr{N} \textrm{ s.t. }(x\in V \textrm{ and } V\subset U)\textrm{ for }(U\in \mathscr{T} \textrm{ and } x\in U))$$
$\mathscr{B}:=\{\mathscr{N}_x\in \mathscr{T}\mid x\in X\}$ is the base for $\mathscr{T}$
\end{defin}

\begin{remark}
Trivial topology is the weakest topology on $X$, and discrete topology the strongest. $\mathscr{T}(\mathscr{E})$ is the weakest topology on $X$ that contains $\mathscr{E}$.
\end{remark}

\begin{proposition}~\\
a. 设拓扑空间$(X,\mathscr{T})$,则$\mathscr{E}\subset \mathscr{T}$为$\mathscr{T}$的基 \textrm{iff} 任一$\mathscr{T}$中开集是$\mathscr{E}$中元素之并 \newline
b. $\mathscr{E}\subset P(X)$为X上一拓扑的基 \textrm{iff} 
\[
(x\in V\textrm{ and }V\in \mathscr{E}\textrm{ for all } x\in X)\wedge(\exists W\in \mathscr{E}\textrm{ s.t. }x\in W\subset (U\cap V)\textrm{ for all }x\in U\cap V\textrm{ and }U,V\in \mathscr{E})
\]
c. 若$\mathscr{E}\subset P(X)$,设所有$\mathscr{E}$中元素有限交的并为S,则$\mathscr{T}(\mathscr{E})=\{\emptyset,X,S\}$
\end{proposition}

\begin{defin}~\\
first countable:$X$中的每一点都有$\mathscr{T}$的可数邻域基 \newline
second countable:$\mathscr{T}$有可数基
\end{defin}

\begin{proposition}~\\ \label{Prop 4.3}
a. 若X为第一可数空间且$A\subset X$,则$x\in \overline{A}$ \textrm{iff} $\exists\{x_j\}\in A$ \textrm{s.t. }$x_j\to x$ \newline
b. 第二可数空间可分
\end{proposition}

\begin{defin}
separation axioms \newline
$T_0$ (Kolmogorov): 对$x\neq y$至少有一个开集包含其中一个点而不包含另一个点 \newline
$T_1$: 对$x\neq y$有一个开集包含其中任一个点而不包含另一个点 \newline
$T_2$ (Hausdorff): $x\neq y$ admits 2 disjoint open sets each containing each \newline
$T_3$ (regular): $T_1$ + closed set A and $x\in A^c$ admits 2 disjoint open sets each containing each \newline
$T_{3\frac{1}{2}}$ (Tychonoff, or completely regular):
$$
T_1\textrm{ + closed set} A\textrm{ and } x\in A^c \textrm{ admits } f\in C(X,[0,1]) \textrm{ s.t. } f(x)=1 \textrm{ and} \left.f\right|_A=0
$$
$T_4$ (normal): $T_1$ + closed sets $A\cap B$ admits  disjoint open sets each containing each
\end{defin}

\begin{proposition} \label{Prop 4.4}
X is a $T_1$ space iff $\{x\}$ is closed for every $x\in X$
\end{proposition}

\begin{remark}
\autoref{Prop 4.4} shows that every normal space is regular (by Urysohn's Lemma we can further deduce that every normal space is Tychonoff, and every Tychonoff space is regular), and every regular space is Hausdorff. In fact, most topological spaces that arises in practice are Hausdorff, or become Hausdorff after simple modifications.
\end{remark}



\subsection{Continuous Maps}
\begin{proposition}~\\
a. 映射在X上连续iff映射在X中每一点连续 \newline
b. 若$\mathscr{T}_Y=\mathscr{T}(\mathscr{E})$,则$f:X\to Y$连续 iff $f^{-1}(V)$ is open in X for every $V\in \mathscr{E}$
\end{proposition}

\begin{defin}~\\
homeomorphism: 双射+正反连续 \newline
embedding: $f:X\to Y$ 非满单射,且当$f(X)$为$Y$的子拓扑空间时$f:X\to f(X)$为同胚映射 \newline
weak topology generated by $\{f_\alpha:X\to Y_\alpha\}_{\alpha \in A}$:
\[
\mathscr{T}(\{f_\alpha\}_{\alpha \in A}):=\{f^{-1}_\alpha(U_\alpha)\mid \alpha \in A \textrm{ and } U_\alpha \subset \mathscr{T}_{Y_\alpha}\}
\]
product topology on the Cartesian product of topological spaces $X=\prod_{\alpha \in A}X_\alpha$:
\[ \mathscr{T}(\{\pi_\alpha\}_{\alpha \in A})=\textrm{weak topology generated by }\{\pi_\alpha:X \to X_\alpha\}_{\alpha \in A}
\]
\end{defin}

\begin{remark}
Weak topology generated by $\{f_\alpha\}_{\alpha \in A}$ is the unique weakest topology on $X$ that makes all $f_\alpha$ continuous. Let $U_\alpha \subset \mathscr{T}_{X_\alpha}$ and $n\in \mathbb{N}$, a base for the product topology is given by
\[
\left\{\bigcap_{j=1}^{n}\pi^{-1}_{\alpha_j}(U_{\alpha_j})\right\} \textrm{ or } \left\{\prod_{\alpha \in A}U_\alpha \mid U_\alpha=X_\alpha \textrm{ if } \alpha \neq \alpha_1,...,\alpha_n\right\} 
\]
\end{remark}

\begin{proposition}~\\
a. 若每个拓扑空间Hausdorff,则积空间Hausdorff\newline
b. $f:Y\to \prod_{\alpha \in A}X_\alpha$连续iff $\pi_\alpha \circ f$对$\alpha \in A$连续 \newline
c. 设函数项数列$\{f_n\}\in X^A$,则$f_n$在积拓扑中收敛于$f$ iff $f_n$逐点收敛于$f$
\end{proposition}

\begin{defin}
uniform norm of $f\in B(X,\mathbb{C})$: $||f||_u:=\sup{\{|f(x)|\mid x\in X\}}$
\end{defin}

\begin{remark}
Uniform metric (or Chebyshev metric) on $B(X,\mathbb{C})$ is taken to be $\rho(f,g)=||f-g||_u$, and convergence with respect to this metric is uniform convergence on $X$. It's also easy to see that $B(X,\mathbb{C})$ is complete under the uniform metric.
\end{remark}

\begin{proposition}~\\ \label{Prop 4.7}
a. Chebyshev度量下$BC(X,\mathbb{C})$是$B(X,\mathbb{C})$的闭子集 \newline
对$T4$空间X以及其中两个不相交的闭集A、B,设$\Delta=\{k2^{-n}\mid n\geq 1 \textrm{ and } 0<k<2^n\}\subset \mathbb{Q}$ \newline
b. $\exists \{U_r\mid r\in \Delta\}\subset \mathscr{T}_X $ s.t. $A\subset U_r\subset B^c$ for all $r\in \Delta$ and $\overline{U}_r\subset U_s$ for $r<s$ \newline
c. (Urysohn's Lemma) $\exists f\in C(X,[0,1])$ s.t. $\left.f\right|_A=0$ and $\left.f\right|_B=0$
\end{proposition}

\begin{remark}
Numbers of the form in $\Delta$ are called dyadic numbers. Construct an inductive argument that uses the normality of $X$ repeatedly to prove \autoref{Prop 4.7} (b). Urysohn's Lemma shows that every normal space is completely regular, and it's not hard to show that every completely regular space is regular.
\end{remark}

\begin{theorem} (The Tietze Extension Theorem) \newline
设X为$T4$空间,$A\subset X$为闭集且$f\in C(A,[a,b])$,则$\exists F\in C(X,[a,b])$ s.t. $\left.F\right|_A=f$
\end{theorem}



\subsection{Nets}
\begin{defin}~\\
directed set:= 预序集+对有向性(任一对元素有上界) \newline
net in set $X$:=  mapping $\langle x_\alpha\rangle_{\alpha \in A}:A\to X$ where $A$ is directed \newline
eventually in E: \newline
frequently in E: \newline
converges to x: \newline
cluster point: \newline
\end{defin}

\begin{proposition}~\\
a. E为拓扑空间X的子集且$x\in X$,则$x\in acc(E)$ iff $\exists \textrm{ E上网 } \langle e_\alpha\rangle$ s.t. $e_\alpha \to x$ \newline
b. 对拓扑空间X和Y,$f\in C(X,Y)$ iff $\langle f(x_\alpha)\rangle$ 收敛于 f(x) for all net $\langle x_\alpha\rangle$收敛于x
\end{proposition}

\begin{defin}
subnet
\end{defin}

\begin{remark}

\end{remark}

\begin{proposition}

\end{proposition}



\subsection{Compact Spaces}



\subsection{Exercises}
2, 3 (度量空间$T4$), 5 (可分度量空间第二可数), 6 (用Sorgenfrey line $\mathbb{R}_l$构造 \autoref{Prop 4.3} (b)逆命题的反例), 11 \\
\newline
15-17, 19, 21, 22, 24, 28, 29 \\
30-34 \\
38, 39, 43 \\
49, 54 \\
59-61, 63, 64 \\
66-70 \\
\end{document}