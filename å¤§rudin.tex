\documentclass[hidelinks]{article}

\usepackage{xeCJK}
\usepackage[a4paper,top=3cm,bottom=3cm,left=3cm,right=3cm,marginparwidth=1.75cm]{geometry}
\usepackage{amsmath,amsthm,amsfonts}
\usepackage[utf8]{inputenc}
\usepackage{amssymb}
\usepackage[mathscr]{eucal}
\usepackage{graphicx}
\usepackage{xcolor}
\usepackage{physics}

\setlength\parindent{0pt}

\newtheoremstyle{dotless}{}{}{\itshape}{}{\bfseries}{}{ }{}

\theoremstyle{definition}
\newtheorem*{defin}{DEF}
\newtheorem*{eg}{EG}

\theoremstyle{remark}
\newtheorem*{remark}{Remark}

\theoremstyle{plain}
\newtheorem*{cor}{Corollary}

\theoremstyle{dotless}
\newtheorem{innercustomthm}{PROP}
\newenvironment{proposition}[1]
  {\renewcommand\theinnercustomthm{#1}\innercustomthm}
  {\endinnercustomthm}
  
\theoremstyle{dotless}
\newtheorem{innercustomex}{EX}
\newenvironment{exercise}[1]
  {\renewcommand\theinnercustomex{#1}\innercustomex}
  {\endinnercustomex}
  
\DeclareMathOperator\supp{supp}
\DeclareMathOperator\vol{vol}

\usepackage{hyperref}
\hypersetup{colorlinks=false}



\begin{document}

\section{Abstract Integration}

\begin{defin}
topology $\tau\in \mathcal{P}(X)$ in $X$:\begin{enumerate}
    \item $\varnothing,X\in\tau$
    \item finite intersection
    \item arbitrary union
\end{enumerate}
topological space, open sets, continuous
\end{defin}

\begin{defin}
$\sigma$-algebra $\mathfrak{M}\in \mathcal{P}(X)$ in $X$:\begin{enumerate}
    \item $X\in\mathfrak{M}$
    \item complement
    \item countable union
\end{enumerate}
measurable space, measurable sets, measurable (开集的原像为可测集)
\end{defin}

\begin{remark}
$\varnothing\in\mathfrak{M}$, closed under difference and countable intersection.
\end{remark}

\begin{proposition}{1.5}
局部连续性和整体连续性定义的等价性
\end{proposition}

\begin{proposition}{1.7}
连续函数复合连续函数为连续函数,连续函数复合可测函数为可测函数
\end{proposition}

\begin{proposition}{1.8}
实可测函数$u,v$的连续映射$h=\Phi(u,v)$可测
\end{proposition}

\begin{remark}
为复可测函数的构造提供基础,同时证明了可测函数的线性性;将``可测"换为``连续"后命题同样成立
\end{remark}

\begin{proposition}{1.9}
复可测函数的若干性质
\end{proposition}

\begin{proposition}{1.10}
$\mathfrak{M}(\mathscr{F})$的存在性,其中$\mathscr{F}\in\mathcal{P}(X)$
\end{proposition}

\begin{defin}
Borel sets ($\sigma$-algebra generated by open sets), Borel functions 
\end{defin}

\begin{remark}
every continuous mapping is Borel measurable
\end{remark}

\begin{proposition}{1.12}
关于Borel sets的若干性质
\end{proposition}

\begin{defin}
upper limit: $\limsup\limits_{n\to\infty}a_n=\inf\limits_{m\geq1}\sup\limits_{k\geq m}a_k$\newline
lower limit: $\liminf\limits_{n\to\infty}a_n=-\lim\sup\limits_{n\to\infty}(-a_n)=\sup\limits_{m\geq1}\inf\limits_{k\geq m}a_k$
\end{defin}

\begin{proposition}{1.14}
实可测函数列的$\sup$和$\limsup$都是可测函数
\end{proposition}

\begin{defin}
simple function: range is a finite subset of $[0,\infty)$
\end{defin}

\begin{remark}
simple function $s:X\to\{\alpha_1,\alpha_2,\dots,\alpha_n\}$ can be written as \[s=\sum_{i=1}^n\alpha_i\chi_{A_i}\]
where $A_i=\{x:s(x)=\alpha_i\}$, and is measurable iff $A_i$s are measurable.
\end{remark}

\begin{proposition}{1.17}
构造单调递增的simple function列pointwise逼近任一给定的可测函数
\end{proposition}

\begin{defin}
(positive) measure: $\mu:\mathscr{M}\to[0,\infty]$ that is countably additive, and $\exists A\in\mathscr{M}:\mu(A)<\infty$\newline
measure space, complex measure
\end{defin}

\begin{proposition}{1.19}
Monotonicity and continuity of measure
\end{proposition}

\begin{remark}
Example 1.20 (c) provides an interesting counterexamples of measure being continuous from above without the finiteness condition.
\end{remark}

\begin{defin}
integral\begin{enumerate}
    \item of measurable simple function $s=\sum\limits_{i=1}^n\alpha_i\chi_{A_i}$:
    \[\int_Es\,d\mu=\sum\limits_{i=1}^n\alpha_i\mu(A_i\cap E)\]
    \item of measurable $f:X\to[0,\infty]$
    \[\int_Ef\,d\mu=\sup\int_Es\,d\mu\]
\end{enumerate}
for $E\in\mathscr{M}$
\end{defin}

\begin{remark}
The assumption $0\cdot\infty=0$ is made so commutativity, associativity, and distributive laws hold in $[0,\infty]$. Some properties are easily verifiable, such as the monotonicity and scalar multiplication.
\end{remark}

\begin{proposition}{1.25}\label{Prop 1.25}
非负可测simple function的积分构成测度;积分的线性可加性
\end{proposition}

\begin{remark}
\hyperref[Prop 1.29]{Prop 1.29} 和 \hyperref[Prop 1.27]{Prop 1.27} 分别推广了 \hyperref[Prop 1.25]{Prop 1.25} 的两个结论
\end{remark}

\begin{proposition}{1.26}[Lebesgue's Monotone Convergence Theorem]
若单调递增的非负可测函数列$\{f_n\}$趋近$f$,则$f$可测且
\[\int_Xf_n\,d\mu\to\int_Xf\,d\mu\]
\end{proposition}

\begin{proposition}{1.27}\label{Prop 1.27}
Countable additivity of integral
\end{proposition}

\begin{proposition}{1.28}[Fatou's Lemma]
对$f_n:X\to[0,\infty]$
\[\int_X\left(\liminf_{n\to\infty}\right)\,d\mu\leq\liminf_{n\to\infty}\int_Xf_n\,d\mu\]
\end{proposition}

\begin{proposition}{1.29}\label{Prop 1.29}
非负可测函数的积分构成测度,且
\[\int_Xg\,d(\int_Ef\,d\mu)=\int_Xgf\,d\mu\]
\end{proposition}

\begin{remark}
The converse of \hyperref[Prop 1.29]{Prop 1.29} is xxx
\end{remark}

\begin{defin}
Lebesgue integrable functions (summable functions):
\[L^1(\mu)=\{\textrm{complex measurable functions }f:\int_X|f|\,d\mu<\infty\}\]
If $f=u+iv\in L^1(\mu)$ where $u$ and $v$ are real measurable functions, then
\[\int_Ef\,d\mu=\int_Eu^+\,d\mu-\int_Eu^-\,d\mu+i\left(\int_Ev^+\,d\mu-\int_Ev^+\,d\mu\right)\]
\end{defin}

\begin{proposition}{1.32}
If $f,g\in L^1(\mu)$ then $\alpha f+\beta g\in L^1(\mu)$ $\forall\alpha,\beta\in\mathbb{C}$, and
\[\int_X(\alpha f+\beta g)\,d\mu=\alpha\int_Xf\,d\mu+\beta\int_Xg\,d\mu\]
\end{proposition}

\begin{proposition}{1.33}
\[f\in L^1(\mu)\Rightarrow\abs{\int_Xf\,d\mu}\leq\int_X|f|\,d\mu\]
\end{proposition}

\begin{proposition}{1.34}[Lebesgue's Dominated Convergence Theorem]
逐点收敛于$f$复可测函数列$f_n$若满足$\exists\ g\in L^1(\mu):\abs{f_n}\leq g$则$f\in L^1(\mu)$且
\[\lim_{n\to\infty}\int_X\abs{f_n-f}\,d\mu=0\]
\end{proposition}

\begin{defin}
$P$ holds almost everywhere on $E\in\mathfrak{M}$: $P$ holds on $E-N$ where $N\subset E$ and $\mu(N)=0$
\end{defin}

\begin{proposition}{1.36}
测度和$\sigma$-algebra的完备化
\end{proposition}

\begin{remark}
With the completion of measure and $sigma$-measure we can relax the definition and condition of many properties above.
\end{remark}

\begin{proposition}{1.38}
若复可测函数列$\{f_n\}$满足
\[\sum_{n=1}^\infty\int_X\abs{f_n}\,d\mu<\infty\]
a.e. on $X$,则级数$\sum\limits_{n=1}^\infty f_n(x)$收敛a.e. on $X$,$f\in L^1(\mu)$,且
\[\int_X\sum_{n=1}^\infty f\,d\mu=\sum_{n=1}^\infty\int_Xf_n\,d\mu\]
\end{proposition}

\begin{proposition}{1.39-1.41}
a.e.-related theorems
\end{proposition}

\begin{exercise}{1.1}
cardinality of the continuum $\mathfrak{c}$
\end{exercise}

\begin{exercise}{1.2}

\end{exercise}

\section{Positive Borel Measures}

\begin{defin}
linear functional: linear transformation s.t. the target space is the field of scalars\newline
positive linear functional: $f\geq0\Rightarrow\Lambda f\geq0$
\end{defin}

\begin{proposition}{2.4}
紧集的闭子集是紧集
\end{proposition}

\begin{proposition}{2.5}
Hausdorff空间中紧集和紧集外一点的可分性
\end{proposition}

\begin{cor}
Hausdorff空间中的紧集是闭集
\end{cor}

\begin{defin}
Finite-Intersection Property:
\end{defin}

\begin{proposition}{2.6}
Hausdorff空间中的紧集有Finite-Intersection Property
\end{proposition}

\begin{proposition}{2.7}
Locally Compact Hausdorff空间中若有紧集$K$和开集$U$满足$K\subset U$,则存在开集$V$及其紧闭包$\overline{V}$s.t. $K\subset V\subset\overline{V}\subset U$
\end{proposition}

\begin{defin}
lower (resp. upper) semicontinuous: $\{x:f(x)>\alpha,\ \forall\alpha\in\mathbb{R}\}$ (resp. $\{x:f(x)<\alpha,\ \forall\alpha\in\mathbb{R}\}$)
\end{defin}

\begin{remark}
Characteristic functions of open sets are lower semicontinuous; the supremum of any collection of lower semicontinuous functions is lower semicontinuous.
\end{remark}

\begin{defin}
support of $f$: $\supp(f)=\overline{A}$ for $A=\{x:f(x)\neq0\}$
\end{defin}

\begin{remark}
The collection of all continuous complex functions on $X$ with compact support $C_c(X)$ forms a vector space over $\mathbb{C}$.
\end{remark}

\begin{proposition}{2.7}
连续函数保紧致性
\end{proposition}

\begin{cor}
$f\in C_c(X)$的值域是复平面上的紧致集
\end{cor}

\begin{defin}
$K\prec f$: $K$ is compact in $X$, $f\in C_c(X)$, $0\leq f\leq1$, and $f_K=1$\newline
$f\prec V$: $V$ is open in $X$, $f\in C_c(X)$, $0\leq f\leq1$, and $\supp(f)\subset V$
\end{defin}

\begin{proposition}{2.12}[Urysohn's Lemma]
In LCH if $K\subset V$ then $\exists f:K\prec f\prec V$
\end{proposition}

\begin{proposition}{2.13}
In LCH if compact set $K$ satisfies $K\subset\bigcup V_i$ then $\exists$ a partition of unity $\{h_i\}$ on $K$ subordinate to the open cover $\{V_i\}$ s.t.\begin{enumerate}
    \item $h_i\prec V_i$
    \item $\forall x\in K:\sum h_i(x)=1$
\end{enumerate}
for $i=1,2,...,n$
\end{proposition}

\begin{proposition}{2.14}[Riesz Representation Theorem]\label{Prop 2.14}
Let $X$ be a LCH, $\Lambda$ a positive linear functional on $C_c(X)$. There exists a $\sigma$-algebra $\mathfrak{M}$ that contains all Borel sets in X, and a unique positive measure $\mu$ on $\mathfrak{M}$ that represents $\Lambda$, i.e. $\forall f\in C_c(X):\Lambda  f=\int_Xf\,d\mu$, satisfying\begin{enumerate}
    \item $\forall\textrm{ compact }K:\mu(K)<\infty$
    \item $\forall E\in\mathfrak{M}:\mu(E)=\inf\{\mu(V):E\subset V\textrm{, V open}\}$ \normalfont{($E$ is outer regular)}
    \item $\forall$ open $E\in\mathfrak{M}$ s.t. $\mu(E)<\infty:\mu(E)=\sup\{\mu(K):K\subset E\textrm{, K compact}\}$ \normalfont{($E$ is inner regular)}
    \item The measure space $(X,\mathfrak{M},\mu)$ is complete
\end{enumerate}
\end{proposition}

\begin{defin}
Borel measure: measure defined on the $\sigma$-algebra of all Borel sets in a LCH\newline
Borel measure is regular: every Borel set is both inner and outer regular\newline
$\sigma$-compact: countable union of compact sets\newline
$\sigma$-finite (w.r.t. $\mu$): countable union of sets with finite measure\newline
$F_\sigma$ (resp. $G_\delta$): countable union (resp. intersection) of closed (resp. open) sets\newline
$F_\sigma$-$G_\delta$ Property: 
\end{defin}

\begin{proposition}{2.17}
If in \hyperref[Prop 2.14]{Prop 2.14} LCH $X$ is $\sigma$-compact, then $\mu$ is regular and has $F_\sigma$-$G_\delta$ Property
\end{proposition}

\begin{proposition}{2.18}
Let $X$ be a LCH in which every open set is $\sigma$-compact, then any positive Borel measure $\lambda$ on $X$ s.t. $\forall$ compact $K:\mu(K)<\infty$ is regular
\end{proposition}

\begin{defin}
$k$-cell: $W=\{x:\alpha_i\sim x_i\sim\beta_i,1\leq i\leq k\}$ where $\sim$ is $\leq$ or $<$\newline
volume of $k$-cell: $\vol(W)=\prod\limits_{i=1}^k(\beta_i-\alpha_1)$\newline
$\delta$-box with corner at $a=(\alpha_1,\alpha_2,\dots,\alpha_k)$: $Q(a,\delta)=\{x:\alpha_i\leq x_i<\alpha_i+\delta,1\leq i\leq k\}$
\end{defin}

\begin{remark}
Following 
\end{remark}

\begin{proposition}{2.20}
存在complete measure space $(\mathbb{R}^k,\mathfrak{M},m)$ \normalfont{(Lebesgue measure on $\mathbb{R}^k$)}满足:\begin{enumerate}
    \item $\forall$ $k$-cell $W:m(W)=\vol(W)$
    \item $\mathfrak{M}\supset\mathfrak{B}(\mathbb{R}^k)$, $m$正则且有$F_\sigma$-$G_\delta$ Property
    \item transnational invariant
    \item 对任意$\mathbb{R}^k$上满足translational invariant的Borel positive measure $\mu$若$\forall$ compact $K:\mu(K)<\infty$则$\exists$ constant $c$ s.t. $\forall E\in\mathfrak{B}(\mathbb{R}^k):\mu(E)=cm(E)$
\end{enumerate}
\end{proposition}

\begin{remark}
A counterexample of Lebesgue measure defined on all subsets of $\mathbb{R}^k$.
\end{remark}



\end{document}