\documentclass[hidelinks]{article}

\usepackage{xeCJK}
\usepackage[a4paper,top=3cm,bottom=3cm,left=3cm,right=3cm,marginparwidth=1.75cm]{geometry}
\usepackage{amsmath,amsthm,amsfonts}
\usepackage[utf8]{inputenc}
\usepackage{amssymb}


\setcounter{section}{-1}

\theoremstyle{definition}
\newtheorem*{defin}{Def}
\theoremstyle{plain}
\newtheorem{theorem}{Thm}[section]
\newtheorem{proposition}[theorem]{Prop}
\theoremstyle{remark}
\newtheorem*{remark}{Remark}

\usepackage{hyperref}
\hypersetup{colorlinks=false}



\begin{document}

\section{Prologue}

\subsection{The Language of Set Theory}
\begin{defin}~\\
limit inferior (resp. superior):
\begin{gather*}
\liminf E_n:=\bigcup_{n=1}^{\infty}\left(\bigcap_{k=n}^{\infty}E_k\right)=\{x\mid x\not\in E_n \textrm{ for finitely many n}\} \\
\limsup E_n:=\bigcap_{n=1}^{\infty}\left(\bigcup_{k=n}^{\infty}E_k\right)=\{x\mid x\in E_n \textrm{ for infinitely many n}\}
\end{gather*}
Cartesian product:
\[
\prod_{\alpha \in A}X_\alpha:=\{f:A\to \bigcup_{\alpha \in A}X_\alpha \mid f(\alpha)\in X_\alpha \textrm{ for every } \alpha \in A\}
\]
$\alpha$th projection (coordinate map): $\pi_\alpha:X\to X_\alpha$ by $\pi_\alpha(f)\mapsto f(\alpha)$
\end{defin}


\subsection{Orderings}
\begin{defin}~\\
preorder (quasi-order):= 自反+传递 \newline
partial ordering:= 自反+反对称+传递 \newline
linear (total) ordering:= partial ordering +完全,or反对称+传递+完全 \newline
posets $X$ and $Y$ are order isomorphic if 
$$\exists f: X\leftrightarrow Y \textrm{ s.t. } x_1\leq x_2 \textrm{ iff } f(x_1)\leq f(x_2)$$
\end{defin}

\begin{theorem}(The Hausdorff Maximal Principle)\label{Thm 0.1}
任一偏序集中存在一极大链,包含偏序集中任一链
\end{theorem}
\begin{theorem}(Zorn's Lemma)\label{Thm 0.2}
任一偏序集中存在一极大元,若任一链在该集中有上界
\end{theorem}

\begin{defin}
well ordering:= linear ordering +任一非空子集存在(唯一的)极小元
\end{defin}

\begin{theorem}(The WOP)\label{Thm 0.3}
任一非空集都可以被良序排序
\end{theorem}
\begin{theorem}(The Axiom of Choice)\label{Thm 0.4}
非空集族上Catersian product非空
\end{theorem}

\begin{remark}
在ZF中 \autoref{Thm 0.1} 与 \autoref{Thm 0.2}、\autoref{Thm 0.3}、\autoref{Thm 0.4} 等价.
\end{remark}

\subsection{Cardinality}
\begin{defin}~\\
$\mathrm{card}(X)\leq \mathrm{card}(Y)$:= $\exists f: X\hookrightarrow Y$ (injective) \newline
$\mathrm{card}(X)=\mathrm{card}(Y)$:= $\exists f: X\leftrightarrow Y$ (bijective) \newline
$\mathrm{card}(X)\geq \mathrm{card}(Y)$:= $\exists f: X\twoheadrightarrow Y$ (surjective)
\end{defin}

\begin{proposition}~\\
a. $\mathrm{card}(X)\leq \mathrm{card}(Y)$ iff $\mathrm{card}(Y)\geq \mathrm{card}(X)$ \newline
b. either $\mathrm{card}(X)\leq \mathrm{card}(Y)$ or $\mathrm{card}(X)\geq \mathrm{card}(Y)$
\end{proposition}
\begin{theorem}(The Schröder-Bernstein Theorem)~\\
若 $\mathrm{card}(X)\leq \mathrm{card}(Y)$ 且 $\mathrm{card}(Y)\leq \mathrm{card}(X)$,则 $\mathrm{card}(X)=\mathrm{card}(Y)$
\end{theorem}

\begin{defin}~\\
$X$ is countable/denumerable:= $\mathrm{card}(X)\leq \mathrm{card}(\mathbb{N})$ \newline
$X$ has the cardinality of the continuum $\mathfrak{c}$:= $\mathrm{card}(X)=\mathrm{card}(\mathbb{R})$
\end{defin}

\begin{theorem}(Cantor's Theorem)\label{Thm 0.7}
对任何集合X及其幂集P(X)存在$\mathrm{card}(X)\leq \mathrm{card}(P(X))$
\end{theorem}

\begin{remark}
The construction of $\{x\in X\mid x\not\in g(x)\}$ in the proof of \autoref{Thm 0.7} is called Cantor's diagonal argument. From \autoref{Thm 0.7} one immediately deduces that $X$ is uncountable if $\mathrm{card}(X)\geq \mathfrak{c}$, the converse of which, namely the continuum hypothesis, remains an open problem.
\end{remark} 

\begin{proposition}~\\
a. $\mathrm{card}(P(\mathbb{N}))=\mathfrak{c}$ \newline
b. 若 $\mathrm{card}(X)\leq \mathrm{card}(\mathbb{N})$ 且 $\mathrm{card}(Y)\leq \mathrm{card}(\mathbb{N})$,则 $\mathrm{card}(X\times Y)\leq \mathrm{card}(\mathbb{N})$ \newline
c. 若$\mathrm{card}(X)\leq \mathrm{card}(\mathbb{R})$ 且 $\mathrm{card}(Y)\leq \mathrm{card}(\mathbb{R})$,则 $\mathrm{card}(X\times Y)\leq \mathrm{card}(\mathbb{R})$ \newline
d. 若 $\mathrm{card}(A)\leq \mathrm{card}(\mathbb{N})$ 且 $\mathrm{card}(X_\alpha)\leq \mathrm{card}(\mathbb{N})$ 对任何 $\alpha \in A$ 成立,则 $\mathrm{card}(\cup_{\alpha \in A}X_\alpha)\leq \mathrm{card}(\mathbb{N})$ \newline
e. 若 $\mathrm{card}(A)\leq \mathrm{card}(\mathbb{R})$ 且 $\mathrm{card}(X_\alpha)\leq \mathrm{card}(\mathbb{R})$ 对任何 $\alpha \in A$ 成立,则 $\mathrm{card}(\cup_{\alpha \in A}X_\alpha)\leq \mathrm{card}(\mathbb{R})$ \newline
f. 对无穷集$X$,$\mathrm{card}(X)\leq \mathrm{card}(\mathbb{N})$ 蕴含 $\mathrm{card}(X)=\mathrm{card}(\mathbb{N})$
\end{proposition}

\begin{remark}
Immediately, $\mathbb{Z}$ and $\mathbb{Q}$ are countable.
\end{remark}

\subsection{More about Well Ordered Sets}
\begin{defin}~\\
initial segment of $x$: $I_x:=\{y\in X\mid y<x\}$ \newline
predecessors of $x$: elements of initial segment $I_x$
\end{defin}

\begin{theorem}(The Principle of Transfinite Induction)~\\
良序集 X 中,若子集 $A\subset X$ 满足 $I_x\subset A\Rightarrow x\in A$,则 A=X
\end{theorem}

\begin{remark}
超限归纳法是数学归纳法从 $\mathbb{N}$ 向任意良序集的推广.
\end{remark}

\begin{proposition}~\\
a. 若X良序且$A\subset X$,则$\bigcup_{x\in A}I_x$ 为X的一前段或本身 \newline
b. X 和 Y良序,则either X与Y序同构,or X与Y的一个前段序同构,or Y与X的一个前段序同构
\end{proposition}

\begin{defin}~\\
set of countable ordinals: 不可数良序集$\Omega$,且$I_x$可数for all $x\in \Omega$ \newline
first uncountable ordinal: $\omega_1:=\sup \Omega$
\end{defin}

\begin{proposition}~\\
a. $\Omega$ 存在且对良序集在序同构意义下唯一 \newline
b. 任一$\Omega$的可数子集有上界 \newline
c. $\mathbb{N}$ 序同构于$\Omega$的一子集
\end{proposition}

\subsection{The Extended Real Number System}
\begin{defin}
For arbitraty set $X$ and $f:X\to [0,\infty]$
$$\sum_{x\in X}f(x):=\sup\left\{\sum_{x\in F}f(x)\mid \textrm{finite} F\subset X\right\}$$
\end{defin}

\begin{remark}
It's often convenient to assume that, unless otherwise stated, $0\cdot (\pm \infty)=0$
\end{remark}

\begin{proposition}~\\
a. 设 $f:X\to [0,\infty]$ 且 $A=\{x\mid f(x)>0\}$,则 $\forall g: \mathbb{N}\leftrightarrow A$
\begin{displaymath}
\sum_{x\in X}f(x) = \left\{ \begin{array}{ll}
\sum\limits_{k=1}^{\infty}f(g(k)) & \textrm{if }\mathrm{card}(A)=\mathrm{card}(\mathbb{N})\\
\infty & \textrm{if A is uncountable}
\end{array} \right.
\end{displaymath}
b. $\mathbb{R}$ 中任一开集可以写成可数个开区间的无交并
\end{proposition}

\subsection{Metric Spaces}
\begin{defin}~\\
metric:= 正定+对称+三角不等式 \newline
product metric on $X_1\times X_2$
$$\rho((x_1, x_2),(y_1, y_2):=max(\rho(x_1,y_1),\rho(x_2,y_2)$$
given metric spaces $(X_1,\rho_1)$ and $(X_2,\rho_2)$ \newline
metrics $\rho_1$ and $\rho_2$ on set $X$ are equivalent if
\[
\exists C,C'>0 \textrm{ s.t. } C\rho_1\leq \rho_2\leq C'\rho_1
\]
$E$ is dense in $X$ if $\overline{E}=X$ \newline
$E$ is nowhere dense if $(\overline{E})^{\mathrm{o}}=0$ \newline
$X$ is separable if it has a countable dense subset
\end{defin}

\begin{remark}
A consequence of the definition of metric equivalence is that the product metric need to be unique, but up to a equivalence class. In fact, most results concerning metric spaces depend not on the particular metric chosen but only on its equivalence class.
\end{remark}

\begin{proposition} \label{Prop 0.10}~\\
a. $x\in \overline{E}$ \textrm{ iff } $\forall r>0 \textrm{ s.t. } B(r,x)\cup E\neq \emptyset$ \textrm{ iff } $\exists \{x_n\}\in E\textrm{ s.t. }\lim_{n\to \infty}x_n=x$ \newline
b. 映射连续当且仅当其逆映射将开集映射到开集 \newline
c. 完备度量空间的闭子集完备,任一度量空间的完备子集为闭集
\end{proposition}

\begin{defin}
E is totally bounded if $\forall \epsilon >0$ E 能被有限个半径为 $\epsilon$ 的开球覆盖 \newline
\end{defin}

\begin{proposition} \label{Prop 0.11}
E 为紧致集 \textrm{iff} E 为列紧集 \textrm{iff} E 完备且完全有界
\end{proposition}

\begin{remark}
紧致又称满足the Heine-Borel Property,列紧又称满足the Bolzano-Weierstrass Property.
\end{remark}

\begin{proposition}(The Heine–Borel theorem)
$\mathbb{R}^n$ 中紧致集与有界闭集等价
\end{proposition}

\begin{remark}
事实上,从定义知完全有界$\Rightarrow$有界,再综合 \autoref{Prop 0.11} 和 \autoref{Prop 0.10} (c) 即得紧致集$\Rightarrow$有界闭集;但逆命题一般不成立. 在 $\mathbb{R}^n$ 中证明两者等价只需论证有界$\Rightarrow$完全有界即可.
\end{remark}


\subsection{Exercises}
none

\end{document}