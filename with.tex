\documentclass[hidelinks,10pt]{article}

\usepackage[margin=1in]{geometry}
\usepackage{amsmath,amsthm,amsfonts}
\usepackage[utf8]{inputenc}
\usepackage{amssymb}
\usepackage[mathscr]{eucal}
\usepackage{graphicx}
\usepackage{listings}
\usepackage{xcolor}
\usepackage[OT1]{fontenc}
\usepackage{physics}
\usepackage{xpatch}
\usepackage{nicefrac}

\setlength\parindent{0pt}

\newtheoremstyle{dotless}{}{}{\itshape}{}{\bfseries}{}{ }{}

\newtheoremstyle{dotles}{}{}{\upshape}{}{\bfseries}{}{ }{}

\theoremstyle{definition}
\newtheorem*{defin}{DEF}

\theoremstyle{dotles}
\newtheorem{innercustomex}{EX}
\newenvironment{exercise}[1]
  {\renewcommand\theinnercustomex{#1}\innercustomex}
  {\endinnercustomex}

\theoremstyle{dotless}
\newtheorem{proposition}{PROP}[section]

\theoremstyle{remark}
\newtheorem*{remark}{Remark}
  
\DeclareMathOperator\supp{supp}
\DeclareMathOperator\vol{vol}
\DeclareMathOperator{\End}{End}
\DeclareMathOperator{\esssup}{ess\,sup}

\usepackage{hyperref}
\hypersetup{colorlinks=false}



\begin{document}
\tableofcontents
\newpage

\section{Abstract Integration}

\subsection{$\sigma$-algebra and measurable space}

\begin{defin}~\\
\textbf{topology} $\tau\in \mathcal{P}(X)$ in $X$:\begin{itemize}
    \item $\varnothing,X\in\tau$
    \item closed under finite intersection
    \item closed under arbitrary union
\end{itemize}
$(X,\tau)$ is a \textbf{topological space}, members $A\in\tau$ are \textbf{open sets}, and functions $f:(X,\tau)\to(Y,\tau')$ s.t. $\forall\,V\in\tau':f^{-1}(V)\in\tau$ are \textbf{continuous}.\bigbreak

\textbf{$\sigma$-algebra} $\mathfrak{M}\in \mathcal{P}(X)$ in $X$:\begin{itemize}
    \item $X\in\mathfrak{M}$
    \item closed under complement
    \item closed under countable union
\end{itemize}
$(X,\mathfrak{M})$ is a \textbf{measurable space}, members $A\in\mathfrak{M}$ are \textbf{measurable sets}, and functions $f:(X,\mathfrak{M})\to(Y,\tau)$ s.t. $\forall\,V\in\tau:f^{-1}(V)\in\mathfrak{M}$ are \textbf{measurable}.\bigbreak

$\mathfrak{M}(\mathscr{F})$, the smallest $\sigma$-algebra that contains $\mathscr{F}\in\mathscr{P}(X)$, is the $\sigma$-algebra \textbf{generated} by $\mathscr{F}$\newline
\textbf{Borel $\sigma$-algebra} $\mathfrak{B}$ of $X$ is  the $\sigma$-algebra generated by open sets in $X$, members of which are \textbf{Borel sets}, and functions $f:(X,\mathfrak{B})\to(Y,\tau)$ s.t. $f^{-1}(V)\in\mathfrak{B},\ \forall\,V\in\tau$ are \textbf{Borel functions}.
\end{defin}

\begin{proposition}\label{Prop 1.1}
Let $u$ and $v$ be real measurable functions and $\Phi$ a continuous mapping on $\mathbb{R}^2$, then $h=\Phi(u,v)$ is measurable.
\end{proposition}

\begin{proposition}\label{Prop 1.2}
Let $f:(X,\mathfrak{M})\to[-\infty,\infty]$. If $\forall\,\alpha\in\mathbb{R}:f^{-1}((\alpha,\infty])\in\mathfrak{M}$, then $f$ is measurable.
\end{proposition}

\subsection{Simple functions and measures}

\begin{defin}~\\
\textbf{upper limit}: $\limsup_{n\to\infty}a_n=\inf_{m\geq1}\sup_{k\geq m}a_k$\newline
\textbf{lower limit}: $\liminf_{n\to\infty}a_n=-\lim\sup_{n\to\infty}(-a_n)=\sup_{m\geq1}\inf_{k\geq m}a_k$\bigbreak

\textbf{simple function}: non-negative function $s:X\to\{\alpha_1,\alpha_2,\dots,\alpha_n\}$ with finite range, often represented by
\[s=\sum_{i=1}^n\alpha_i\chi_{A_i}\]
where $A_i=\{x:s(x)=\alpha_i\}$\bigbreak

\textbf{positive measure} or \textbf{measure}: countably additive function $\mu:\mathfrak{M}\to[0,\infty]$ s.t. $\exists A\in\mathfrak{M}:\mu(A)<\infty$\newline
$(X,\mathfrak{M},\mu)$ is a \textbf{measure space}, and countably additive complex function $\nu$ on $\mathfrak{M}$ is a \textbf{complex measure}.
\end{defin}

\begin{proposition}
Let $\{f_n\}$ be a sequence of extended-real measurable functions, then $g=\sup_{n\geq1}f_n$ and $h=\limsup_{n\to\infty}f_n$ are measurable.
\end{proposition}

\begin{proposition}
Let $f$ be an non-negative extended-real measurable function, then $exists$ a monotonically increasing sequence of simple functions $\{s_n\}$ that pointwise converges to $f$
\end{proposition}

\begin{proposition}
positive measure $\mu$ satisfies:\begin{itemize}
    \item $\mu(\varnothing)=0$
    \item finite additivity and monotonicity
    \item \normalfont{(continuous from below)} $\mu(A_n)\to\mu(\bigcup_{n\geq1}A_n)$ for monotonically increasing sequence $A_n$
    \item \normalfont{(continuous from above)} $\mu(A_n)\to\mu(\bigcap_{n\geq1}A_n)$ for monotonically decreasing sequence $A_n$ if $\mu(A_1)<\infty$
\end{itemize}
\end{proposition}

\subsection{Integration}

\begin{defin}
\textbf{integral}\begin{itemize}
    \item of measurable simple function $s=\sum_{i=1}^n\alpha_i\chi_{A_i}$:
    \[\int_Es\,d\mu=\sum\limits_{i=1}^n\alpha_i\mu(A_i\cap E)\]
    \item of measurable $f:X\to[0,\infty]$
    \[\int_Ef\,d\mu=\sup\int_Es\,d\mu\]
    \item of complex measurable function $h=u+iv\in L^1(\mu)$ where $u$ and $v$ are real measurable functions:
    \[\int_Ef\,d\mu=\int_Eu^+\,d\mu-\int_Eu^-\,d\mu+i\left(\int_Ev^+\,d\mu-\int_Ev^+\,d\mu\right)\]
\end{itemize}
where $E\in\mathfrak{M}$\bigbreak
Members of
\[L^1(\mu)=\{\textrm{complex measurable functions }f:\int_X|f|\,d\mu<\infty\}\]
are \textbf{Lebesgue integrable functions} or \textbf{summable functions}
\end{defin}

\begin{remark}
The assumption $0\cdot\infty=0$ is made so that commutativity, associativity, and distributive laws hold in $[0,\infty]$.
\end{remark}

\begin{proposition}[Lebesgue's Monotone Convergence Theorem]
Let $\{f_n:X\to[0,\infty]\}$ be a monotonically increasing sequence of measurable functions, then
\[\lim_{n\to\infty}\int_Xf_n\,d\mu=\int_X\lim_{n\to\infty}f_n\,d\mu\]
\end{proposition}

\begin{proposition}\label{Prop 1.7}
Let $f:(X,\mathfrak{M},\mu)\to[0,\infty]$ be measurable, then\begin{itemize}
    \item If $E,F\in\mathfrak{M}$, $E\subset F$, $f\leq g$, and $c\in[0,\infty)$ then
    \[\int_Ef\,d\mu\leq\int_Eg\,d\mu,\int_Ef\,d\mu\leq\int_Ff\,d\mu,\int_Ecf\,d\mu=c\int_Ef\,d\mu\]
    \item If $f=\sum_{n\geq1}f_n$ then
    \[\int_Xf\,d\mu=\sum_{n\geq1}\int_Xf_n\,d\mu\]
    \item If $\forall\,E\in\mathfrak{M}:\phi(E)=\int_Ef\,d\mu$ then $\phi$ is a measure on $\mathfrak{M}$ and
    \[\int_Xg\,d\phi=\int_Xgf\,d\mu\]
    for all measurable $g:X\to[0,\infty]$
\end{itemize}
\end{proposition}

\begin{proposition}[Fatou's Lemma]
Let $\{f_n:X\to[0,\infty]\}$ be a sequence of measurable functions, then
\[\int_X(\liminf_{n\to\infty}f_n)\,d\mu\leq\liminf_{n\to\infty}\int_Xf_n\,d\mu\]
\end{proposition}

\begin{proposition}
Let $f,g\in L^1(\mu)$ and $\alpha,\beta\in\mathbb{C}$, then\begin{itemize}
    \item \[\abs{\int_Xf\,d\mu}\leq\int_X\abs{f}\,d\mu\]
    \item $\alpha f+\beta g\in L^1(\mu)$ and
    \[\int_X(\alpha f+\beta g)\,d\mu=\alpha\int_Xf\,d\mu+\beta\int_Xg\,d\mu\]
\end{itemize}
\end{proposition}

\begin{proposition}[Lebesgue's Dominated Convergence Theorem]
Let $\{f_n\}$ be a sequence of complex measurable functions on $X$ with pointwise limit $f$, then if $\exists\,g\in L^1(\mu):\abs{f_n}\leq g$ then $f\in L^1(\mu)$ and
\[\lim_{n\to\infty}\int_Xf_n\,d\mu=\int_Xf\,d\mu\]
\end{proposition}

\subsection{Complete measure and a.e.}

\begin{defin}
P holds \textbf{almost everywhere} (\textbf{a.e.}) on $E\in\mathfrak{M}$: $P$ holds on $E-N$ where $N\subset E$ and $\mu(N)=0$
\end{defin}

\begin{proposition}
Let $(X,\mathfrak{M},\mu)$ be a measure space and the \textbf{$\mu$-completion} of $\mathfrak{M}$
\[\overline{\mathfrak{M}}=\{E\in\mathscr{P}(X):\exists A,B\in\mathfrak{M}\textrm{ s.t. }A\subset E\subset B\textrm{ and }\mu(B-A)=0\}\]
Define in this case $\mu(E)=\mu(A)$, then $(X,\overline{\mathfrak{M}},\mu)$ is a measure space that is \textbf{complete}.
\end{proposition}

\begin{proposition}
The following statements regarding a.e. property holds:\begin{itemize}
    \item Let $\{f_n\in L^1(\mu)\}$ be defined a.e. on $X$ and
    \[\sum_{n=1}^\infty\int_X|f_n|\,d\mu<\infty\]
    then $f(x)=\sum_{n=1}^\infty f_n(x)\in L^1(\mu)$, converges a.e. on $X$, and
    \[\int_Xf\,d\mu=\sum_{n=1}^\infty\int_Xf_n\,d\mu\]
    \item If $f\in L^1(\mu)$ and
    \[\abs{\int_Xf\,d\mu}=\int_X\abs{f}\,d\mu\]
    then $\exists$ constant $\alpha:\alpha f=\abs{f}$ a.e. on $X$
    \item If $\{E_k\}$ is a sequence of measurable sets in $X$ s.t.
    \[\sum_{k=1}^\infty\mu(E_k)<\infty\]
    then $x$ lies in at most finitely many $E_k$s a.e. on $X$
    \item Suppose $\mu(X)<\infty$ and $f\in L^1(\mu)$. If $\forall\,E\in\mathfrak{M}$ with positive measure the average
    \[A_E(f)=\frac{1}{\mu(E)}\int_Ef\,d\mu\]
    lies in some closed set $S\subset\mathbb{C}$, then $f(x)\in S$ a.e. on $X$
\end{itemize}
\end{proposition}

\subsection*{Exercises}

\begin{exercise}{1.1}
cardinality of the continuum $\mathfrak{c}$
\end{exercise}

\begin{exercise}{1.2}
2 $\rightarrow$ finite generalization of \hyperref[Prop 1.1]{Prop 1.1}
\end{exercise}

\begin{exercise}{1.3}
real $\rightarrow$ rational generalization of \hyperref[Prop 1.2]{Prop 1.2}
\end{exercise}

\begin{exercise}{1.4}
limit superior/inferior
\end{exercise}

\begin{exercise}{1.5}
for (b) use Cauchy sequence to characterize
\end{exercise}

\begin{exercise}{1.6}
assume the measurable functions are real-valued
\end{exercise}

\begin{exercise}{1.7}
use Lebesgue's Dominated Convergence Theorem
\end{exercise}

\begin{exercise}{1.8}
strict equality of Fatou's Lemma
\end{exercise}

\begin{exercise}{1.9}

\end{exercise}

\begin{exercise}{1.10}

\end{exercise}

\begin{exercise}{1.11}
very nice construction!
\end{exercise}

\begin{exercise}{1.12}
use \hyperref[Prop 1.7]{Prop 1.7}
\end{exercise}

\begin{exercise}{1.13}
$\psi(E_k)=\int_X\chi_{E_k}\,d\psi=\int_X\chi_{E_k}\abs{f}\,d\mu\to0$, since $\mu(E_k)\to0$
\end{exercise}

\begin{remark}
Redo 1.4(b), 1.5, 1.6, 1.9, 1.10
\end{remark}

\newpage

\section{Positive Borel Measures}

\subsection{Topological prerequisites}

\begin{defin}~\\
\textbf{support} of $f$: $\supp(f)=\overline{A}$ for $A=\{x:f(x)\neq0\}$\newline
$C_c(X)$: the vector space formed by complex continuous functions on $X$ whose support is compact\newline
A complex function $f$ on $X$ \textbf{vanishes at infinity} if $\forall\,\epsilon>0\ \exists$ compact $K\subset X\ \forall\,x\not\in K:\abs{f(x)}<\epsilon$\newline
$C_0(X)$: the class of functions on $X$ that vanishes at infinity\bigbreak
$K\prec f$: $K$ is compact in $X$, $f\in C_c(X)$, $0\leq f\leq1$, and $f_K=1$\newline
$f\prec V$: $V$ is open in $X$, $f\in C_c(X)$, $0\leq f\leq1$, and $\supp(f)\subset V$\bigbreak
$F_\sigma$ set: countable union of closed sets\newline
$G_\delta$ set: countable intersection of open sets
\end{defin}

\begin{proposition}The following compactness properties hold:
\begin{itemize}
    \item Closed subsets of compact sets are compact
    \item In Hausdorff space compact sets are closed
    \item In locally compact Hausdorff space if $K\subset U$ for compact $K$ and open $U$, then $\exists$ open $V$ with compact closure s.t. $K\subset V\subset\overline{V}\subset U$
    \item Continuous mapping maps compact sets to compact sets
\end{itemize}
\end{proposition}

\begin{proposition}[Urysohn's Lemma] In locally compact Hausdroff space if $K\subset V$ for compact $K$ and open $V$, then $\exists f\in C_c(X):K\prec f\prec V$
\end{proposition}

\begin{proposition}In locally compact Hausdorff space if compact set $K$ satisfies $K\subset\bigcup V_i$ then $\exists$ a \textbf{partition of unity} $\{h_i\}$ on $K$ subordinate to the open cover $\{V_i\}$ s.t.\begin{itemize}
    \item $h_i\prec V_i$
    \item $\forall x\in K:\sum h_i(x)=1$
\end{itemize}
for $i=1,2,...,n$
\end{proposition}

\subsection{Riesz Representation Theorem, regularity, and the construction of Lebesgue measure}

\begin{defin}~\\
\textbf{linear functional}: linear transformation whose target space is the field of scalars\newline
A linear function $\Lambda$ is \textbf{positive} if $f\geq0\Rightarrow\Lambda f\geq0$\bigbreak
\textbf{lower semicontinuous}: $\{x:f(x)>\alpha,\ \forall\alpha\in\mathbb{R}\}$ is open\newline
\textbf{upper semicontinuous}: $\{x:f(x)<\alpha,\ \forall\alpha\in\mathbb{R}\}$ is open\bigbreak
\textbf{outer regular}: $\mu(E)=\inf\{\mu(V):E\subset V\textrm{ and }V\textrm{ open}\}$\newline
\textbf{inner regular}: $\mu(E)=\inf\{\mu(K):E\subset K\textrm{ and }E\textrm{ compact}\}$\newline
A measure is \textbf{regular} if every measurable set is both inner and outer regular.\bigbreak
\textbf{Borel measure}: measure defined on the $\sigma$-algebra of all Borel sets in a locally compact Hausdorff space\bigbreak
A set is \textbf{$\sigma$-compact} if it is a countable union of compact sets.\newline
A set has \textbf{$\sigma$-finite} measure if it is a countable union of finite-measure sets.
\end{defin}

\begin{proposition}[Riesz Representation Theorem]\label{Prop 2.4}Let $X$ be a locally compact Hausdorff space and $\Lambda$ a positive linear functional on $C_c(X)$, then $\exists$ a $\sigma$-algebra $\mathfrak{M}$ that contains all Borel sets in X, and a unique positive measure $\mu$ on $\mathfrak{M}$ that represents $\Lambda$, i.e. $\forall f\in C_c(X):\Lambda  f=\int_Xf\,d\mu$, satisfying\begin{itemize}
    \item $\forall\textrm{ compact }K:\mu(K)<\infty$
    \item $\forall E\in\mathfrak{M}:E\textrm{ is outer regular}$
    \item $\forall$ open $E\in\mathfrak{M}$ s.t. $\mu(E)<\infty:E\textrm{ is inner regular}$
    \item The measure space $(X,\mathfrak{M},\mu)$ is complete
\end{itemize}
\end{proposition}

\begin{proposition}Let $X$, $\mu$, and $\mathfrak{M}$ be as \normalfont{\hyperref[Prop 2.4]{Prop 2.4}}\textit{ then}
\begin{itemize}
\item \textit{If $X$ is $\sigma$-compact, then $\mu$ is regular and $\forall\,E\in\mathfrak{M}\ \exists\,F_\sigma,G_\delta$ s.t. $F_\sigma\subset E\subset G_\delta$ and $\mu(G_\delta-F_\sigma)=0$}
\item \textit{If further that every open set in $X$ is $\sigma$-compact, then any positive Borel measure $\lambda$ on $X$ is regular if $\forall$ compact $K:\lambda(K)<\infty$}
\end{itemize}
\end{proposition}

\begin{proposition}There exists a positive complete measure $($\textbf{Lebesgue measure}$)$ $m$ defined on a $\sigma$-algebra $\mathfrak{M}$ in $\mathbb{R}^k$ satisfying:\begin{itemize}
    \item $m(W)=\prod_{i=1}^k(\beta_i-\alpha_i)$ for every $W=\{(x_1,x_2,\dots,x_k):\alpha_i<x_i<\beta_i,\,1\leq i\leq k\}$
    \item $\mathfrak{M}$ contains all Borel sets of $\mathbb{R}^k$
    \item $m$ is translation-invariant, i.e. $\forall\,E\in\mathfrak{M}\ \forall\, x\in\mathbb{R}^k:m(E+x)=m(E)$, and $\forall$ positive translation-invariant Borel measure $\mu$ on $\mathbb{R}^k$ s.t. $\mu(K)$ is finite for every compact $K$, $\exists$ constant $c$ s.t. $\mu=cm$
    \item To every linear transformation $T\in\End(\mathbb{R}^k)$ corresponds $\Delta(T)\in\mathbb{R}$ s.t. $m(T(E))=\Delta(T)m(E)$ for all $E\in\mathfrak{M}$. In particular $\Delta(T)=1$ if $T$ is a rotation
\end{itemize}
\end{proposition}

\subsection{Approximation of measurable functions by continuous functions}

\begin{proposition}[Lusin's Theorem]Let $X$ be a locally compact Hausdorff space and $\mu$ the Lebesgue measure. Let $f$ be a complex measurable function on $X$ s.t. $f|_{A^c}=0$ for $\mu(A)<\infty$, then
\[\forall\,\epsilon\ \exists\,g\in C_c(X):\mu(\{x:f(x)\neq g(x)\})<\epsilon\]
We could further arrange that $\sup_{x\in X}|g(x)|\leq\sup_{x\in X}|f(x)|$
\end{proposition}

\begin{proposition}[Vitali-Carathéodory Theorem]Let $X$ be a locally compact Hausdorff space and real-valued $f\in L^1(\mu)$, then $\forall\,\epsilon\ \exists\,u,v$ s.t. $u\leq f\leq v$, $u$ is upper semicontinuous and bounded above, $v$ is lower semicontinuous and bounded below, and
\[\int_X(v-u)\,d\mu<\epsilon\]
\end{proposition}

\subsection*{Exercises}

\begin{exercise}{2.1}
characteristic functions of open/closed sets
\end{exercise}

\begin{exercise}{2.2}
$\psi(x)=\lim_{\delta\to0}\psi(x,\delta)$
\end{exercise}

\begin{exercise}{2.3}
stupid inequality skill
\end{exercise}

\begin{remark}
Redo: 2.1, 2.2, 2.3
\end{remark}

\newpage

\section{$L^p$-Spaces}

\subsection{Convexity and basic inequalities}

\begin{defin}~\\
A real function $\psi$ is \textbf{convex} on $(a,b)\in\overline{\mathbb{R}}$ if $\forall\,x,y\in(a,b)\ \forall\,\lambda\in[0,1]:\psi((1-\lambda)x+\lambda y\leq(1-\lambda)\psi(x)+\lambda\psi(y)$\bigbreak
\textbf{conjugate exponent}: positive extended-real numbers s.t. $p+q=pq$
\end{defin}

\begin{proposition}
The following statements regarding the convexity of $\psi$ on $(a.b)$ holds:\begin{itemize}
    \item Graphically a point $(t,\psi(t))$ on the curve lies on or below the line connecting $(x,\psi(x))$ and $(y,\psi(y))$ for $a<x<t<y<b$
    \item $\forall\,a<x<t<y<b:(u-t)(\psi(t)-\psi(s))\leq(t-s)(\psi(u)-\psi(t))$
    \item $\psi$ is continuous on $(a,b)$
\end{itemize}
\end{proposition}

\begin{proposition}[Jensen's Inequality]
Let $(\Omega,\mathfrak{M},\mu)$ be a measure space with $\mu(\Omega)=1$. If $\psi$ is convex on $(a,b)\subset\overline{\mathbb{R}}$ and real-valued $f\in L^1(\mu)$ satisfies $a<f<b$ on $\Omega$ then
\[\psi\left(\int_\Omega f\,d\mu\right)\leq\int_\Omega(\psi\circ f)\,d\mu\]
\end{proposition}

\begin{proposition}[Hölder's Inequality]
Let $p,q$ be conjugate exponents and $1<p<\infty$ and $f,g$ be non-negative measurable functions on measure space $(X,\mathfrak{M},\mu)$, then
\[\int_Xfg\,d\mu\leq\bigg\{\int_Xf^p\,d\mu\bigg\}^{\nicefrac{1}{p}}\bigg\{\int_Xg^q\,d\mu\bigg\}^{\nicefrac{1}{q}}\]
\end{proposition}

\begin{proposition}[Minkowski's Inequality]
Let $f,g$ be non-negative measurable functions on measure space $(X,\mathfrak{M},\mu)$ and $1<p<\infty$, then
\[\bigg\{\int_X(f+g)^p\,d\mu\bigg\}^{\nicefrac{1}{p}}\leq\bigg\{\int_Xf^p\,d\mu\bigg\}^{\nicefrac{1}{p}}+\bigg\{\int_Xg^p\,d\mu\bigg\}^{\nicefrac{1}{p}}\]
\end{proposition}

\subsection{The $L^p$-spaces}

\begin{defin}~\\
Let $f:X\to[0,\infty]$ be measurable and $S=\{\alpha:\mu(f^{-1}((\alpha,\infty]))=0\}$, the \textbf{essential supremum} of $f$ is
\[\esssup f=\begin{cases}
\inf S,&S\neq\varnothing\\
\infty,&S=\varnothing
\end{cases}\]\bigbreak
Let $f$ be a complex measurable function on $X$, the \textbf{$L^p$-norm} of $f$ is\begin{itemize}
    \item $\norm{f}_p=\big\{\int_X\abs{f}^p\,d\mu\big\}^{\nicefrac{1}{p}}$, if $0<p<\infty$
    \item $\norm{f}_\infty=\esssup\abs{f}$
\end{itemize}
$L^p$-space is $L^p(\mu)=\{\textrm{complex measurable functions }f:\norm{f}_p<\infty\}$, the members of $L^\infty(\mu)$ are \textbf{essentially bounded}.
\end{defin}

\begin{proposition}
Let $X$ be a Locally compact Hausdorff space, then\begin{itemize}
    \item $L^p(\mu)$ is the completion of $C_c(X)$ w.r.t. the $L^p$-norm metric, if $1\leq p<\infty$
    \item $C_0(X)$ is the completion of $C_c(X)$ w.r.t the supremum-norm metric defined by $\norm{f}=\sup_{x\in X}\abs{f(x)}$
\end{itemize}
\end{proposition}

\subsection*{Exercises}

\begin{exercise}{3.1}
r
\end{exercise}

\newpage

\section{Elementary Hilbert Space Theory}

\subsection{Inner product and Hilbert space}

\begin{defin}~\\
Let $H$ be a complex vector space, the \textbf{inner product} $(x,y)$ for $x,y,z\in H$ is a complex number that satisfies:\begin{itemize}
    \item $(y,x)=\overline{(x,y)}$
    \item $(\alpha x+y,z)=\alpha[(x,z)+(y,z)]$
    \item $(x,x)\geq0$, the equality holds only if $x=0$
\end{itemize}
Space $H$ endowed with inner product is a \textbf{inner product space}.\bigbreak
\textbf{norm} of $x\in H$: $\norm{x}=(x,x)^{\nicefrac{1}{2}}$\bigbreak
A set $E$ in a vector space $V$ is \textbf{convex} if $\forall\,x,y\in E\ \forall\,t\in(0,1):z=(1-t)x+ty\in E$\newline
$x$ is \textbf{orthogonal} to $y$, or $x\perp y$, if $(x,y)=0$, the set of all $y\in H$ that are orthogonal to $x$ is $x^\perp$, and the set of all $y\in H$ that are orthogonal to every $x\in M$ for some $M\subset H$ is $M^\perp$
\end{defin}

\begin{proposition}
Every nonempty, closed, convex set $E$ in a Hilbert space $H$ contains a unique element of smallest norm. 
\end{proposition}

\begin{proposition}
Let $M$ be a closed subspace of a Hilbert space $H$, then $\exists$ \textbf{orthogonal projections} $P,Q$ of $H$ onto $M$ and $M^\perp$ satisfying\begin{itemize}
    \item $\forall\,x\in H\ \exists!$ decomposition $x=Px+Qx$ s.t. $Px\in M$ and $Qx\in M^\perp$, hence $\norm{x}^2=\norm{Px}^2+\norm{Qx}^2$
    \item $P:H\to M$ and $Q:H\to M^\perp$ are linear functionals
    \item $Px$ and $Qx$ are the nearest points to $x$ in $M$ and $M^\perp$ resp.
\end{itemize}
\end{proposition}

\begin{proposition}
All continuous linear functionals on a Hilbert space $H$ are of the type $L_y:x\mapsto(x,y)$
\end{proposition}

\subsection{Orthonormality and trigonometric system}



\end{document}